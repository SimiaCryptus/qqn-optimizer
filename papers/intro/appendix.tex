\hypertarget{appendix-a-problem-family-vs-optimizer-family-comparison-matrix}{%
\section{Appendix A: Problem Family vs Optimizer Family Comparison Matrix}\label{appendix-a-problem-family-vs-optimizer-family-comparison-matrix}}

\documentclass{article}
\usepackage{booktabs}
\usepackage{array}
\usepackage{multirow}
\usepackage{xcolor}
\usepackage{colortbl}
\usepackage{siunitx}
\usepackage{adjustbox}
\usepackage{rotating}
\usepackage{makecell}
\usepackage[margin=0.5in]{geometry}
\usepackage{longtable}
\definecolor{bestgreen}{RGB}{0,150,0}
\definecolor{worstred}{RGB}{200,0,0}
\begin{document}
\begin{longtable}{lccccc}
\caption{Optimizer Family vs Problem Family Performance Matrix}
\label{tab:family_vs_family_matrix} \\
\toprule
\makecell{Problem\\Family} & \makecell{\rotatebox{90}{\textbf{Adam}}} & \makecell{\rotatebox{90}{\textbf{GD}}} & \makecell{\rotatebox{90}{\textbf{L-BFGS}}} & \makecell{\rotatebox{90}{\textbf{QQN}}} & \makecell{\rotatebox{90}{\textbf{Trust Region}}} \\
\midrule
\endfirsthead
\multicolumn{6}{l}{\tablename\ \thetable\ -- continued from previous page} \\
\toprule
\makecell{Problem\\Family} & \makecell{\rotatebox{90}{\textbf{Adam}}} & \makecell{\rotatebox{90}{\textbf{GD}}} & \makecell{\rotatebox{90}{\textbf{L-BFGS}}} & \makecell{\rotatebox{90}{\textbf{QQN}}} & \makecell{\rotatebox{90}{\textbf{Trust Region}}} \\
\midrule
\endhead
\midrule
\multicolumn{6}{l}{Continued on next page} \\
\endfoot
\bottomrule
\endlastfoot
\textbf{Ackley} &  \makecell{16.3 / 12.0 \\ \scriptsize{Adam-AMSGrad} \\ \scriptsize{Adam-Fast}}&  \makecell{16.5 / 9.7 \\ \scriptsize{GD} \\ \scriptsize{GD-Momentum}}&  \makecell{6.5 / 2.3 \\ \scriptsize{L-BFGS} \\ \scriptsize{Conservative}}& \cellcolor{bestgreen!30} \makecell{5.1 / 1.0 \\ \scriptsize{Bisection-1} \\ \scriptsize{GoldenSection}}& \cellcolor{worstred!20} \makecell{20.7 / 13.7 \\ \scriptsize{Conservative} \\ \scriptsize{Aggressive}} \\
\textbf{Barrier} &  \makecell{16.3 / 10.7 \\ \scriptsize{WeightDecay} \\ \scriptsize{Adam-Fast}}&  \makecell{13.8 / 8.0 \\ \scriptsize{GD} \\ \scriptsize{AdaptiveMom...}}&  \makecell{6.5 / 2.7 \\ \scriptsize{Aggressive} \\ \scriptsize{Conservative}}& \cellcolor{bestgreen!30} \makecell{4.3 / 1.3 \\ \scriptsize{Bisection-2} \\ \scriptsize{CubicQuadIn...}}& \cellcolor{worstred!20} \makecell{19.1 / 14.0 \\ \scriptsize{Conservative} \\ \scriptsize{Aggressive}} \\
\textbf{Beale} & \cellcolor{worstred!20} \makecell{19.0 / 8.0 \\ \scriptsize{WeightDecay} \\ \scriptsize{Adam-Fast}}& \cellcolor{bestgreen!30} \makecell{8.8 / 3.0 \\ \scriptsize{GD-Nesterov} \\ \scriptsize{GD-Momentum}}&  \makecell{10.0 / 2.0 \\ \scriptsize{MoreThuente} \\ \scriptsize{Aggressive}}&  \makecell{8.8 / 1.0 \\ \scriptsize{GoldenSection} \\ \scriptsize{Bisection-2}}&  \makecell{18.4 / 15.0 \\ \scriptsize{Precise} \\ \scriptsize{Standard}} \\
\textbf{Booth} & \cellcolor{worstred!20} \makecell{19.2 / 11.0 \\ \scriptsize{WeightDecay} \\ \scriptsize{Adam-Robust}}&  \makecell{14.6 / 10.0 \\ \scriptsize{GD} \\ \scriptsize{GD-Momentum}}&  \makecell{11.0 / 6.0 \\ \scriptsize{MoreThuente} \\ \scriptsize{Aggressive}}& \cellcolor{bestgreen!30} \makecell{3.0 / 1.0 \\ \scriptsize{CubicQuadIn...} \\ \scriptsize{GoldenSection}}&  \makecell{17.2 / 12.0 \\ \scriptsize{Adaptive} \\ \scriptsize{Conservative}} \\
\textbf{GoldsteinPrice} &  \makecell{13.2 / 10.0 \\ \scriptsize{Adam-AMSGrad} \\ \scriptsize{Adam-Fast}}&  \makecell{15.2 / 9.0 \\ \scriptsize{GD-Momentum} \\ \scriptsize{GD}}&  \makecell{10.4 / 5.0 \\ \scriptsize{MoreThuente} \\ \scriptsize{Aggressive}}& \cellcolor{bestgreen!30} \makecell{3.2 / 1.0 \\ \scriptsize{GoldenSection} \\ \scriptsize{Bisection-2}}& \cellcolor{worstred!20} \makecell{23.0 / 21.0 \\ \scriptsize{Aggressive} \\ \scriptsize{Precise}} \\
\textbf{Griewank} &  \makecell{17.7 / 12.0 \\ \scriptsize{Adam-Fast} \\ \scriptsize{Adam-Robust}}&  \makecell{12.0 / 7.7 \\ \scriptsize{GD-Momentum} \\ \scriptsize{GD}}&  \makecell{7.9 / 3.7 \\ \scriptsize{Aggressive} \\ \scriptsize{L-BFGS-Limited}}& \cellcolor{bestgreen!30} \makecell{6.3 / 1.0 \\ \scriptsize{StrongWolfe} \\ \scriptsize{CubicQuadIn...}}& \cellcolor{worstred!20} \makecell{21.1 / 13.7 \\ \scriptsize{Conservative} \\ \scriptsize{Aggressive}} \\
\textbf{Himmelblau} & \cellcolor{worstred!20} \makecell{18.8 / 11.0 \\ \scriptsize{WeightDecay} \\ \scriptsize{Adam-Robust}}&  \makecell{14.6 / 9.0 \\ \scriptsize{GD} \\ \scriptsize{AdaptiveMom...}}&  \makecell{11.2 / 5.0 \\ \scriptsize{L-BFGS-Limited} \\ \scriptsize{Aggressive}}& \cellcolor{bestgreen!30} \makecell{3.4 / 1.0 \\ \scriptsize{GoldenSection} \\ \scriptsize{Bisection-1}}&  \makecell{17.0 / 8.0 \\ \scriptsize{Adaptive} \\ \scriptsize{Conservative}} \\
\textbf{IllConditionedRosenbrock} &  \makecell{14.2 / 9.0 \\ \scriptsize{WeightDecay} \\ \scriptsize{Adam-Robust}}&  \makecell{12.5 / 7.0 \\ \scriptsize{GD-Nesterov} \\ \scriptsize{GD-Momentum}}&  \makecell{12.5 / 4.7 \\ \scriptsize{MoreThuente} \\ \scriptsize{Aggressive}}& \cellcolor{bestgreen!30} \makecell{4.1 / 1.7 \\ \scriptsize{CubicQuadIn...} \\ \scriptsize{GoldenSection}}& \cellcolor{worstred!20} \makecell{21.8 / 16.7 \\ \scriptsize{Aggressive} \\ \scriptsize{Conservative}} \\
\textbf{Levi} &  \makecell{14.4 / 11.0 \\ \scriptsize{Adam-Robust} \\ \scriptsize{Adam-Fast}}&  \makecell{14.6 / 9.0 \\ \scriptsize{GD-Momentum} \\ \scriptsize{GD-WeightDecay}}&  \makecell{11.6 / 3.0 \\ \scriptsize{L-BFGS-Limited} \\ \scriptsize{Aggressive}}& \cellcolor{bestgreen!30} \makecell{3.8 / 1.0 \\ \scriptsize{GoldenSection} \\ \scriptsize{Bisection-1}}& \cellcolor{worstred!20} \makecell{20.6 / 13.0 \\ \scriptsize{Conservative} \\ \scriptsize{Aggressive}} \\
\textbf{Levy} &  \makecell{15.9 / 10.0 \\ \scriptsize{WeightDecay} \\ \scriptsize{Adam-AMSGrad}}&  \makecell{16.2 / 7.7 \\ \scriptsize{GD-WeightDecay} \\ \scriptsize{AdaptiveMom...}}&  \makecell{9.1 / 6.7 \\ \scriptsize{Conservative} \\ \scriptsize{Aggressive}}& \cellcolor{bestgreen!30} \makecell{3.0 / 1.0 \\ \scriptsize{Bisection-2} \\ \scriptsize{StrongWolfe}}& \cellcolor{worstred!20} \makecell{20.8 / 16.0 \\ \scriptsize{Conservative} \\ \scriptsize{Aggressive}} \\
\textbf{Matyas} &  \makecell{13.2 / 10.0 \\ \scriptsize{Adam-Fast} \\ \scriptsize{Adam-AMSGrad}}&  \makecell{16.0 / 12.0 \\ \scriptsize{GD-Momentum} \\ \scriptsize{AdaptiveMom...}}&  \makecell{8.8 / 3.0 \\ \scriptsize{L-BFGS} \\ \scriptsize{Aggressive}}& \cellcolor{bestgreen!30} \makecell{4.0 / 1.0 \\ \scriptsize{StrongWolfe} \\ \scriptsize{Bisection-1}}& \cellcolor{worstred!20} \makecell{23.0 / 21.0 \\ \scriptsize{Conservative} \\ \scriptsize{Precise}} \\
\textbf{Michalewicz} & \cellcolor{bestgreen!30} \makecell{6.2 / 1.0 \\ \scriptsize{Adam-Fast} \\ \scriptsize{Adam-Robust}}&  \makecell{12.1 / 6.7 \\ \scriptsize{AdaptiveMom...} \\ \scriptsize{GD-WeightDecay}}&  \makecell{14.3 / 7.0 \\ \scriptsize{MoreThuente} \\ \scriptsize{Aggressive}}&  \makecell{11.9 / 6.7 \\ \scriptsize{Bisection-2} \\ \scriptsize{CubicQuadIn...}}& \cellcolor{worstred!20} \makecell{20.5 / 16.3 \\ \scriptsize{Conservative} \\ \scriptsize{Aggressive}} \\
\textbf{Neural Networks} &  \makecell{9.2 / 2.5 \\ \scriptsize{WeightDecay} \\ \scriptsize{Adam-Robust}}&  \makecell{19.6 / 16.0 \\ \scriptsize{GD-WeightDecay} \\ \scriptsize{AdaptiveMom...}}&  \makecell{11.0 / 8.0 \\ \scriptsize{Conservative} \\ \scriptsize{MoreThuente}}& \cellcolor{bestgreen!30} \makecell{3.8 / 1.0 \\ \scriptsize{CubicQuadIn...} \\ \scriptsize{StrongWolfe}}& \cellcolor{worstred!20} \makecell{21.4 / 18.5 \\ \scriptsize{Adaptive} \\ \scriptsize{Aggressive}} \\
\textbf{NoisySphere} &  \makecell{16.9 / 8.7 \\ \scriptsize{Adam-Fast} \\ \scriptsize{Adam}}&  \makecell{8.8 / 5.3 \\ \scriptsize{GD-Nesterov} \\ \scriptsize{GD-WeightDecay}}& \cellcolor{bestgreen!30} \makecell{7.4 / 1.0 \\ \scriptsize{Conservative} \\ \scriptsize{Aggressive}}&  \makecell{9.9 / 2.7 \\ \scriptsize{StrongWolfe} \\ \scriptsize{CubicQuadIn...}}& \cellcolor{worstred!20} \makecell{19.9 / 16.3 \\ \scriptsize{Conservative} \\ \scriptsize{Precise}} \\
\textbf{PenaltyI} &  \makecell{8.1 / 4.3 \\ \scriptsize{Adam-AMSGrad} \\ \scriptsize{Adam-Fast}}&  \makecell{12.3 / 9.7 \\ \scriptsize{GD} \\ \scriptsize{GD-Nesterov}}&  \makecell{14.3 / 5.7 \\ \scriptsize{Conservative} \\ \scriptsize{Aggressive}}& \cellcolor{bestgreen!30} \makecell{7.5 / 1.0 \\ \scriptsize{CubicQuadIn...} \\ \scriptsize{Bisection-2}}& \cellcolor{worstred!20} \makecell{22.9 / 20.7 \\ \scriptsize{Adaptive} \\ \scriptsize{Precise}} \\
\textbf{Rastrigin} &  \makecell{11.4 / 4.7 \\ \scriptsize{Adam-AMSGrad} \\ \scriptsize{Adam-Fast}}&  \makecell{14.2 / 7.7 \\ \scriptsize{GD-WeightDecay} \\ \scriptsize{GD-Momentum}}&  \makecell{14.1 / 3.7 \\ \scriptsize{MoreThuente} \\ \scriptsize{Aggressive}}& \cellcolor{bestgreen!30} \makecell{9.9 / 3.0 \\ \scriptsize{CubicQuadIn...} \\ \scriptsize{Bisection-2}}& \cellcolor{worstred!20} \makecell{15.4 / 7.0 \\ \scriptsize{Adaptive} \\ \scriptsize{Conservative}} \\
\textbf{Regression} &  \makecell{18.5 / 13.2 \\ \scriptsize{Adam-Fast} \\ \scriptsize{Adam-Robust}}&  \makecell{13.6 / 8.2 \\ \scriptsize{AdaptiveMom...} \\ \scriptsize{GD-Momentum}}&  \makecell{8.9 / 4.8 \\ \scriptsize{Conservative} \\ \scriptsize{L-BFGS}}& \cellcolor{bestgreen!30} \makecell{3.4 / 1.0 \\ \scriptsize{Bisection-1} \\ \scriptsize{GoldenSection}}& \cellcolor{worstred!20} \makecell{20.6 / 17.2 \\ \scriptsize{Adaptive} \\ \scriptsize{Conservative}} \\
\textbf{Rosenbrock} &  \makecell{13.4 / 6.0 \\ \scriptsize{Adam-Fast} \\ \scriptsize{Adam-Robust}}&  \makecell{12.1 / 5.0 \\ \scriptsize{GD-Nesterov} \\ \scriptsize{GD-Momentum}}&  \makecell{12.6 / 4.0 \\ \scriptsize{MoreThuente} \\ \scriptsize{Aggressive}}& \cellcolor{bestgreen!30} \makecell{4.9 / 2.0 \\ \scriptsize{StrongWolfe} \\ \scriptsize{GoldenSection}}& \cellcolor{worstred!20} \makecell{22.0 / 17.7 \\ \scriptsize{Aggressive} \\ \scriptsize{Conservative}} \\
\textbf{SVM} &  \makecell{13.5 / 8.5 \\ \scriptsize{WeightDecay} \\ \scriptsize{Adam-Fast}}&  \makecell{13.9 / 5.0 \\ \scriptsize{GD-WeightDecay} \\ \scriptsize{AdaptiveMom...}}&  \makecell{9.6 / 3.0 \\ \scriptsize{Conservative} \\ \scriptsize{L-BFGS-Limited}}& \cellcolor{bestgreen!30} \makecell{6.3 / 2.5 \\ \scriptsize{StrongWolfe} \\ \scriptsize{GoldenSection}}& \cellcolor{worstred!20} \makecell{21.7 / 17.0 \\ \scriptsize{Conservative} \\ \scriptsize{Aggressive}} \\
\textbf{Schwefel} & \cellcolor{worstred!20} \makecell{20.1 / 10.7 \\ \scriptsize{Adam-Fast} \\ \scriptsize{Adam-Robust}}&  \makecell{10.5 / 6.7 \\ \scriptsize{GD-WeightDecay} \\ \scriptsize{GD}}&  \makecell{10.5 / 4.3 \\ \scriptsize{Conservative} \\ \scriptsize{Aggressive}}& \cellcolor{bestgreen!30} \makecell{4.3 / 1.0 \\ \scriptsize{StrongWolfe} \\ \scriptsize{GoldenSection}}&  \makecell{19.7 / 16.7 \\ \scriptsize{Standard} \\ \scriptsize{Conservative}} \\
\textbf{SparseQuadratic} &  \makecell{18.9 / 12.5 \\ \scriptsize{WeightDecay} \\ \scriptsize{Adam-AMSGrad}}&  \makecell{14.5 / 10.5 \\ \scriptsize{GD-WeightDecay} \\ \scriptsize{AdaptiveMom...}}&  \makecell{6.4 / 1.5 \\ \scriptsize{MoreThuente} \\ \scriptsize{L-BFGS}}& \cellcolor{bestgreen!30} \makecell{4.9 / 1.5 \\ \scriptsize{GoldenSection} \\ \scriptsize{Bisection-1}}& \cellcolor{worstred!20} \makecell{20.3 / 14.5 \\ \scriptsize{Precise} \\ \scriptsize{Aggressive}} \\
\textbf{SparseRosenbrock} &  \makecell{12.9 / 8.0 \\ \scriptsize{Adam-Fast} \\ \scriptsize{Adam-Robust}}&  \makecell{11.8 / 6.0 \\ \scriptsize{GD-Nesterov} \\ \scriptsize{GD-Momentum}}&  \makecell{15.2 / 4.5 \\ \scriptsize{MoreThuente} \\ \scriptsize{Aggressive}}& \cellcolor{bestgreen!30} \makecell{3.7 / 1.0 \\ \scriptsize{CubicQuadIn...} \\ \scriptsize{Bisection-2}}& \cellcolor{worstred!20} \makecell{21.4 / 19.0 \\ \scriptsize{Standard} \\ \scriptsize{Conservative}} \\
\textbf{Sphere} & \cellcolor{worstred!20} \makecell{20.1 / 14.5 \\ \scriptsize{WeightDecay} \\ \scriptsize{Adam-AMSGrad}}&  \makecell{13.9 / 10.0 \\ \scriptsize{GD-Momentum} \\ \scriptsize{AdaptiveMom...}}&  \makecell{6.1 / 1.0 \\ \scriptsize{Aggressive} \\ \scriptsize{Conservative}}& \cellcolor{bestgreen!30} \makecell{5.3 / 3.0 \\ \scriptsize{StrongWolfe} \\ \scriptsize{GoldenSection}}&  \makecell{19.6 / 14.0 \\ \scriptsize{Conservative} \\ \scriptsize{Aggressive}} \\
\textbf{StyblinskiTang} & \cellcolor{worstred!20} \makecell{16.6 / 4.3 \\ \scriptsize{WeightDecay} \\ \scriptsize{Adam-Robust}}&  \makecell{14.2 / 7.3 \\ \scriptsize{GD-WeightDecay} \\ \scriptsize{AdaptiveMom...}}&  \makecell{10.3 / 2.3 \\ \scriptsize{Conservative} \\ \scriptsize{Aggressive}}& \cellcolor{bestgreen!30} \makecell{8.7 / 1.7 \\ \scriptsize{GoldenSection} \\ \scriptsize{StrongWolfe}}&  \makecell{15.3 / 4.3 \\ \scriptsize{Standard} \\ \scriptsize{Conservative}} \\
\textbf{Trigonometric} &  \makecell{12.7 / 7.3 \\ \scriptsize{Adam} \\ \scriptsize{Adam-Fast}}&  \makecell{14.3 / 5.0 \\ \scriptsize{GD} \\ \scriptsize{GD-Momentum}}&  \makecell{12.7 / 5.3 \\ \scriptsize{MoreThuente} \\ \scriptsize{Aggressive}}& \cellcolor{bestgreen!30} \makecell{4.0 / 1.0 \\ \scriptsize{CubicQuadIn...} \\ \scriptsize{Bisection-2}}& \cellcolor{worstred!20} \makecell{21.3 / 17.7 \\ \scriptsize{Precise} \\ \scriptsize{Aggressive}} \\
\textbf{Zakharov} &  \makecell{13.4 / 9.3 \\ \scriptsize{WeightDecay} \\ \scriptsize{Adam-Robust}}&  \makecell{14.7 / 7.3 \\ \scriptsize{GD} \\ \scriptsize{AdaptiveMom...}}&  \makecell{11.7 / 6.0 \\ \scriptsize{MoreThuente} \\ \scriptsize{L-BFGS}}& \cellcolor{bestgreen!30} \makecell{3.0 / 1.0 \\ \scriptsize{Bisection-1} \\ \scriptsize{StrongWolfe}}& \cellcolor{worstred!20} \makecell{22.2 / 19.0 \\ \scriptsize{Adaptive} \\ \scriptsize{Conservative}} \\
\end{longtable}
\vspace{0.5em}
\textbf{Legend:} Each cell contains:
\begin{itemize}
\item \textbf{Top line:} Average Ranking / Best Rank Average (lower is better)
\item \textbf{Middle line:} Best performing variant in this optimizer family
\item \textbf{Bottom line:} Worst performing variant in this optimizer family
\end{itemize}
\textcolor{bestgreen}{Green cells} indicate the best performing optimizer family for that problem family.
\textcolor{worstred}{Red cells} indicate the worst performing optimizer family.
\end{document}


\hypertarget{appendix-b-theoretical-foundations-and-proofs}{%
\section{Appendix B: Theoretical Foundations and Proofs}\label{appendix-b-theoretical-foundations-and-proofs}}

\hypertarget{b.1-algorithm-derivation}{%
\subsection{B.1 Algorithm Derivation}\label{b.1-algorithm-derivation}}

\hypertarget{b.1.1-the-direction-combination-problem}{%
\subsubsection{B.1.1 The Direction Combination Problem}\label{b.1.1-the-direction-combination-problem}}

Consider the fundamental problem of combining multiple optimization directions. Given:
- Gradient direction: \(-\nabla f(\mathbf{x})\) providing guaranteed descent
- Quasi-Newton direction: \(\mathbf{d}_{\text{QN}}\) offering potential superlinear convergence

We seek a principled method to combine these directions that:

\begin{enumerate}
\def\labelenumi{\arabic{enumi}.}
\tightlist
\item
  Guarantees descent from any starting point
\item
  Smoothly interpolates between the directions
\item
  Requires no additional hyperparameters
\item
  Maintains computational efficiency
\end{enumerate}

\hypertarget{b.1.2-geometric-formulation}{%
\subsubsection{B.1.2 Geometric Formulation}\label{b.1.2-geometric-formulation}}

We formulate direction combination as a boundary value problem in parametric space. Consider a parametric curve \(\mathbf{d}: [0,1] \rightarrow \mathbb{R}^n\) satisfying:

\begin{enumerate}
\def\labelenumi{\arabic{enumi}.}
\tightlist
\item
  \textbf{Initial position}: \(\mathbf{d}(0) = \mathbf{0}\)
\item
  \textbf{Initial tangent}: \(\mathbf{d}'(0) = -\nabla f(\mathbf{x})\) (ensures descent)
\item
  \textbf{Terminal position}: \(\mathbf{d}(1) = \mathbf{d}_{\text{L-BFGS}}\)
\end{enumerate}

The minimal polynomial satisfying these constraints is quadratic:
\[\mathbf{d}(t) = \mathbf{a}t^2 + \mathbf{b}t + \mathbf{c}\]

Applying boundary conditions:

\begin{itemize}
\tightlist
\item
  From condition 1: \(\mathbf{c} = \mathbf{0}\)
\item
  From condition 2: \(\mathbf{b} = -\nabla f(\mathbf{x})\)
\item
  From condition 3: \(\mathbf{a} + \mathbf{b} = \mathbf{d}_{\text{L-BFGS}}\)
\end{itemize}

Therefore: \(\mathbf{a} = \mathbf{d}_{\text{L-BFGS}} + \nabla f(\mathbf{x})\)

This yields the canonical QQN path:
\[\mathbf{d}(t) = t(1-t)(-\nabla f) + t^2 \mathbf{d}_{\text{L-BFGS}}\]

\hypertarget{b.2-convergence-analysis}{%
\subsection{B.2 Convergence Analysis}\label{b.2-convergence-analysis}}

\hypertarget{b.2.1-universal-descent-property}{%
\subsubsection{B.2.1 Universal Descent Property}\label{b.2.1-universal-descent-property}}

\textbf{Lemma B.1} (Universal Descent): For any direction \(\mathbf{d}_{\text{L-BFGS}} \in \mathbb{R}^n\), the QQN path satisfies:
\[\mathbf{d}'(0) = -\nabla f(\mathbf{x})\]

\emph{Proof}: Direct differentiation of \(\mathbf{d}(t) = t(1-t)(-\nabla f) + t^2 \mathbf{d}_{\text{L-BFGS}}\) gives:
\[\mathbf{d}'(t) = (1-2t)(-\nabla f) + 2t\mathbf{d}_{\text{L-BFGS}}\]

Evaluating at \(t=0\): \(\mathbf{d}'(0) = -\nabla f(\mathbf{x})\). \(\square\)
\textbf{Theorem B.1} (Descent Property): For any \(\mathbf{d}_{\text{L-BFGS}}\), there exists \(\bar{t} > 0\) such that \(\phi(t) = f(\mathbf{x} + \mathbf{d}(t))\) satisfies \(\phi(t) < \phi(0)\) for all \(t \in (0, \bar{t}]\).

\emph{Proof}: Since \(\mathbf{d}'(0) = -\nabla f(\mathbf{x})\):
\[\phi'(0) = \nabla f(\mathbf{x})^T(-\nabla f(\mathbf{x})) = -\|\nabla f(\mathbf{x})\|^2 < 0\]

By continuity of \(\phi'\) (assuming \(f\) is continuously differentiable), there exists \(\bar{t} > 0\) such that \(\phi'(t) < 0\) for all \(t \in (0, \bar{t}]\). By the fundamental theorem of calculus:
\[\phi(t) - \phi(0) = \int_0^t \phi'(s) ds < 0\]

for all \(t \in (0, \bar{t}]\). \(\square\)

\hypertarget{b.2.2-global-convergence-analysis}{%
\subsubsection{B.2.2 Global Convergence Analysis}\label{b.2.2-global-convergence-analysis}}

\textbf{Theorem B.2} (Global Convergence): Under standard assumptions:

\begin{enumerate}
\def\labelenumi{\arabic{enumi}.}
\tightlist
\item
  \(f: \mathbb{R}^n \rightarrow \mathbb{R}\) is continuously differentiable
\item
  \(f\) is bounded below: \(f(\mathbf{x}) \geq f_{\text{inf}} > -\infty\)
\item
  \(\nabla f\) is Lipschitz continuous with constant \(L > 0\)
\item
  The univariate optimization finds a point satisfying the Armijo condition
\end{enumerate}

QQN generates iterates satisfying:
\[\liminf_{k \to \infty} \|\nabla f(\mathbf{x}_k)\| = 0\]

\emph{Proof}: We establish convergence through a descent lemma approach.

\textbf{Step 1: Monotonic Decrease}

By Theorem B.1, each iteration produces \(f(\mathbf{x}_{k+1}) < f(\mathbf{x}_k)\) whenever \(\nabla f(\mathbf{x}_k) \neq \mathbf{0}\).

\textbf{Step 2: Sufficient Decrease}

Define \(\phi_k(t) = f(\mathbf{x}_k + \mathbf{d}_k(t))\). Since \(\phi_k'(0) = -\|\nabla f(\mathbf{x}_k)\|^2 < 0\), by the Armijo condition, there exists \(c_1 \in (0, 1)\) and \(\bar{t} > 0\) such that:
\[\phi_k(t) \leq \phi_k(0) + c_1 t \phi_k'(0) = f(\mathbf{x}_k) - c_1 t \|\nabla f(\mathbf{x}_k)\|^2\]

for all \(t \in (0, \bar{t}]\).

\textbf{Step 3: Quantifying Decrease}

Using the descent lemma with Lipschitz constant \(L\):
\[f(\mathbf{x}_{k+1}) \leq f(\mathbf{x}_k) + \nabla f(\mathbf{x}_k)^T \mathbf{d}_k(t_k^*) + \frac{L}{2}\|\mathbf{d}_k(t_k^*)\|^2\]

For the quadratic path with \(t_k^* \in (0, \bar{t}]\):
\[\|\mathbf{d}_k(t)\|^2 = \|t(1-t)(-\nabla f(\mathbf{x}_k)) + t^2\mathbf{d}_{\text{L-BFGS}}\|^2\]

\[\leq 2t^2(1-t)^2\|\nabla f(\mathbf{x}_k)\|^2 + 2t^4\|\mathbf{d}_{\text{L-BFGS}}\|^2\]

For small \(t\), the gradient term dominates, giving:
\[f(\mathbf{x}_k) - f(\mathbf{x}_{k+1}) \geq c\|\nabla f(\mathbf{x}_k)\|^2\]

for some \(c > 0\) independent of \(k\).

\textbf{Step 4: Summability}

Since \(f\) is bounded below and decreases monotonically:
\[\sum_{k=0}^{\infty} [f(\mathbf{x}_k) - f(\mathbf{x}_{k+1})] = f(\mathbf{x}_0) - \lim_{k \to \infty} f(\mathbf{x}_k) < \infty\]

Combined with Step 3:
\[\sum_{k=0}^{\infty} \|\nabla f(\mathbf{x}_k)\|^2 < \infty\]

\textbf{Step 5: Conclusion}
The summability of \(\|\nabla f(\mathbf{x}_k)\|^2\) implies \(\liminf_{k \to \infty} \|\nabla f(\mathbf{x}_k)\| = 0\). \(\square\)

\hypertarget{b.2.3-local-superlinear-convergence}{%
\subsubsection{B.2.3 Local Superlinear Convergence}\label{b.2.3-local-superlinear-convergence}}

\textbf{Theorem B.3} (Local Superlinear Convergence): Let \(\mathbf{x}^*\) be a local minimum with \(\nabla f(\mathbf{x}^*) = \mathbf{0}\) and \(\nabla^2 f(\mathbf{x}^*) = H^* \succ 0\). Assume:

\begin{enumerate}
\def\labelenumi{\arabic{enumi}.}
\tightlist
\item
  \(\nabla^2 f\) is Lipschitz continuous in a neighborhood of \(\mathbf{x}^*\)
\item
  The L-BFGS approximation satisfies the Dennis-Moré condition:
\end{enumerate}

\[\lim_{k \to \infty} \frac{\|(\mathbf{H}_k - (H^*)^{-1})(\mathbf{x}_{k+1} - \mathbf{x}_k)\|}{\|\mathbf{x}_{k+1} - \mathbf{x}_k\|} = 0\]

Then QQN converges superlinearly: \(\|\mathbf{x}_{k+1} - \mathbf{x}^*\| = o(\|\mathbf{x}_k - \mathbf{x}^*\|)\).

\emph{Proof}: We analyze the behavior near the optimum.

\textbf{Step 1: Neighborhood Properties}

By continuity of \(\nabla^2 f\), there exists a neighborhood \(\mathcal{N}\) of \(\mathbf{x}^*\) and constants \(0 < \mu \leq L\) such that:
\[\mu \mathbf{I} \preceq \nabla^2 f(\mathbf{x}) \preceq L \mathbf{I}, \quad \forall \mathbf{x} \in \mathcal{N}\]

\textbf{Step 2: Optimal Parameter Analysis}

Define \(\phi(t) = f(\mathbf{x}_k + \mathbf{d}(t))\) where \(\mathbf{d}(t) = t(1-t)(-\nabla f(\mathbf{x}_k)) + t^2\mathbf{d}_{\text{L-BFGS}}\).
The first derivative is:
\[\phi'(t) = \nabla f(\mathbf{x}_k + \mathbf{d}(t))^T[(1-2t)(-\nabla f(\mathbf{x}_k)) + 2t\mathbf{d}_{\text{L-BFGS}}]\]

The second derivative is:
\[\phi''(t) = [(1-2t)(-\nabla f(\mathbf{x}_k)) + 2t\mathbf{d}_{\text{L-BFGS}}]^T \nabla^2 f(\mathbf{x}_k + \mathbf{d}(t))[(1-2t)(-\nabla f(\mathbf{x}_k)) + 2t\mathbf{d}_{\text{L-BFGS}}]\]

\[+ \nabla f(\mathbf{x}_k + \mathbf{d}(t))^T[-2(-\nabla f(\mathbf{x}_k)) + 2\mathbf{d}_{\text{L-BFGS}}]\]

At \(t = 1\):
\[\phi'(1) = \nabla f(\mathbf{x}_k + \mathbf{d}_{\text{L-BFGS}})^T \mathbf{d}_{\text{L-BFGS}}\]

Using Taylor expansion:
\[\nabla f(\mathbf{x}_k + \mathbf{d}_{\text{L-BFGS}}) = \nabla f(\mathbf{x}_k) + \nabla^2 f(\mathbf{x}_k)\mathbf{d}_{\text{L-BFGS}} + O(\|\mathbf{d}_{\text{L-BFGS}}\|^2)\]

Since \(\mathbf{d}_{\text{L-BFGS}} = -\mathbf{H}_k\nabla f(\mathbf{x}_k)\):
\[\nabla f(\mathbf{x}_k + \mathbf{d}_{\text{L-BFGS}}) = [\mathbf{I} - \nabla^2 f(\mathbf{x}_k)\mathbf{H}_k]\nabla f(\mathbf{x}_k) + O(\|\nabla f(\mathbf{x}_k)\|^2)\]

By the Dennis-Moré condition, as \(k \to \infty\):
\[\|\mathbf{I} - \nabla^2 f(\mathbf{x}_k)\mathbf{H}_k\| \to 0\]

Therefore:
\[\phi'(1) = o(\|\nabla f(\mathbf{x}_k)\|^2)\]

\textbf{Step 3: Optimal Parameter Convergence}

Since \(\phi'(0) = -\|\nabla f(\mathbf{x}_k)\|^2 < 0\) and \(\phi'(1) = o(\|\nabla f(\mathbf{x}_k)\|^2)\), by the intermediate value theorem and the fact that \(\phi\) is strongly convex near \(t = 1\) (due to positive definite Hessian), the minimizer satisfies:
\[t_k^* = 1 + o(1)\]

\textbf{Step 4: Convergence Rate}

With \(t_k^* = 1 + o(1)\):
\[\mathbf{x}_{k+1} = \mathbf{x}_k + \mathbf{d}(t_k^*) = \mathbf{x}_k + (1 + o(1))\mathbf{d}_{\text{L-BFGS}} + o(\|\mathbf{d}_{\text{L-BFGS}}\|)\]

\[= \mathbf{x}_k - \mathbf{H}_k\nabla f(\mathbf{x}_k) + o(\|\nabla f(\mathbf{x}_k)\|)\]

By standard quasi-Newton theory with the Dennis-Moré condition:

\[\|\mathbf{x}_{k+1} - \mathbf{x}^*\| = o(\|\mathbf{x}_k - \mathbf{x}^*\|)\]

establishing superlinear convergence. \(\square\)

\hypertarget{b.3-robustness-analysis}{%
\subsection{B.3 Robustness Analysis}\label{b.3-robustness-analysis}}

\hypertarget{b.3.1-graceful-degradation}{%
\subsubsection{B.3.1 Graceful Degradation}\label{b.3.1-graceful-degradation}}

\textbf{Theorem B.4} (Graceful Degradation): Let \(\theta_k\) be the angle between \(-\nabla f(\mathbf{x}_k)\) and \(\mathbf{d}_{\text{L-BFGS}}\). If \(\theta_k > \pi/2\) (obtuse angle), then the optimal parameter satisfies \(t^* \in [0, 1/2]\), ensuring gradient-dominated steps.

\emph{Proof}: When \(\theta_k > \pi/2\), we have \(\nabla f(\mathbf{x}_k)^T \mathbf{d}_{\text{L-BFGS}} > 0\).

The derivative of our objective along the path is:
\[\frac{d}{dt}f(\mathbf{x}_k + \mathbf{d}(t)) = \nabla f(\mathbf{x}_k + \mathbf{d}(t))^T \mathbf{d}'(t)\]

At \(t = 1/2\):
\[\mathbf{d}'(1/2) = -\frac{1}{2}\nabla f(\mathbf{x}_k) + \mathbf{d}_{\text{L-BFGS}}\]

For small steps from \(\mathbf{x}_k\):
\[\nabla f(\mathbf{x}_k + \mathbf{d}(1/2)) \approx \nabla f(\mathbf{x}_k)\]

Therefore:
\[\left.\frac{d}{dt}f(\mathbf{x}_k + \mathbf{d}(t))\right|_{t=1/2} \approx \nabla f(\mathbf{x}_k)^T[-\frac{1}{2}\nabla f(\mathbf{x}_k) + \mathbf{d}_{\text{L-BFGS}}]\]

\[= -\frac{1}{2}\|\nabla f(\mathbf{x}_k)\|^2 + \nabla f(\mathbf{x}_k)^T\mathbf{d}_{\text{L-BFGS}} > 0\]

when \(\nabla f(\mathbf{x}_k)^T\mathbf{d}_{\text{L-BFGS}} > \frac{1}{2}\|\nabla f(\mathbf{x}_k)\|^2\).

This implies the function increases beyond \(t = 1/2\), so the univariate optimization will find \(t^* \leq 1/2\), giving:

\[\mathbf{x}_{k+1} \approx \mathbf{x}_k + t^*(1-t^*)(-\nabla f(\mathbf{x}_k))\]

Since \(t^* \leq 1/2\), we have \(t^*(1-t^*) \geq t^*(1/2)\), ensuring a gradient-dominated step. \(\square\)

\hypertarget{b.3.2-stability-under-numerical-errors}{%
\subsubsection{B.3.2 Stability Under Numerical Errors}\label{b.3.2-stability-under-numerical-errors}}

\textbf{Theorem B.5} (Numerical Stability): Let \(\tilde{\mathbf{d}}_{\text{L-BFGS}} = \mathbf{d}_{\text{L-BFGS}} + \boldsymbol{\epsilon}\) where \(\boldsymbol{\epsilon}\) represents numerical errors with \(\|\boldsymbol{\epsilon}\| \leq \delta\). The perturbed QQN path:
\[\tilde{\mathbf{d}}(t) = t(1-t)(-\nabla f) + t^2 \tilde{\mathbf{d}}_{\text{L-BFGS}}\]

satisfies:
\[\|\tilde{\mathbf{d}}(t) - \mathbf{d}(t)\| \leq t^2\delta\]

\emph{Proof}: Direct computation:

\[\|\tilde{\mathbf{d}}(t) - \mathbf{d}(t)\| = \|t^2(\tilde{\mathbf{d}}_{\text{L-BFGS}} - \mathbf{d}_{\text{L-BFGS}})\| = t^2\|\boldsymbol{\epsilon}\| \leq t^2\delta\]

For small \(t\) (near the initial descent phase), the error is \(O(t^2\delta)\), providing quadratic error suppression. \(\square\)

\hypertarget{b.4-computational-complexity}{%
\subsection{B.4 Computational Complexity}\label{b.4-computational-complexity}}

\textbf{Theorem B.6} (Computational Complexity): Each QQN iteration requires:

\begin{itemize}
\tightlist
\item
  \(O(n)\) operations for path construction
\item
  \(O(mn)\) operations for L-BFGS direction computation
\item
  \(O(k)\) function evaluations for univariate optimization
\end{itemize}

where \(n\) is the dimension, \(m\) is the L-BFGS memory size, and \(k\) is typically small (3-10).

\emph{Proof}:

\begin{enumerate}
\def\labelenumi{\arabic{enumi}.}
\tightlist
\item
  \textbf{Path construction}: Computing \(\mathbf{d}(t) = t(1-t)(-\nabla f) + t^2 \mathbf{d}_{\text{L-BFGS}}\) requires \(O(n)\) operations for vector arithmetic.
\item
  \textbf{L-BFGS direction}: The two-loop recursion requires \(O(mn)\) operations to compute \(\mathbf{H}_k\nabla f(\mathbf{x}_k)\).
\item
  \textbf{Line search}: Each function evaluation along the path requires \(O(n)\) operations to compute \(\mathbf{x}_k + \mathbf{d}(t)\), plus the cost of evaluating \(f\).
\end{enumerate}

Total complexity per iteration: \(O(mn + kn) + k \cdot \text{cost}(f)\). \(\square\)

\hypertarget{b.5-extensions-and-variants}{%
\subsection{B.5 Extensions and Variants}\label{b.5-extensions-and-variants}}

\hypertarget{b.5.1-gradient-scaling}{%
\subsubsection{B.5.1 Gradient Scaling}\label{b.5.1-gradient-scaling}}

The basic QQN formulation can be enhanced with gradient scaling:
\[\mathbf{d}(t) = t(1-t)\alpha(-\nabla f) + t^2 \mathbf{d}_{\text{L-BFGS}}\]

where \(\alpha > 0\) is a scaling factor.

\textbf{Proposition B.1} (Scaling Invariance): The set of points reachable by the QQN path is invariant to the choice of \(\alpha\). Only the parametrization changes.

\emph{Proof}: Consider the mapping \(s = \beta(t)\) where \(\beta\) is chosen such that:
\[t(1-t)\alpha(-\nabla f) + t^2 \mathbf{d}_{\text{L-BFGS}} = s(1-s)(-\nabla f) + s^2 \mathbf{d}_{\text{L-BFGS}}\]

This gives a bijection between parametrizations, showing that any point reachable with one \(\alpha\) is reachable with another. \(\square\)

\hypertarget{b.5.2-cubic-extension-with-momentum}{%
\subsubsection{B.5.2 Cubic Extension with Momentum}\label{b.5.2-cubic-extension-with-momentum}}

Incorporating momentum leads to cubic interpolation:
\[\mathbf{d}(t) = t(1-t)(1-2t)\mathbf{m} + t(1-t)\alpha(-\nabla f) + t^2 \mathbf{d}_{\text{L-BFGS}}\]

where \(\mathbf{m}\) is the momentum vector.

This satisfies:

\begin{itemize}
\tightlist
\item
  \(\mathbf{d}(0) = \mathbf{0}\)
\item
  \(\mathbf{d}'(0) = \alpha(-\nabla f) + \mathbf{m}\)
\item
  \(\mathbf{d}(1) = \mathbf{d}_{\text{L-BFGS}}\)
\item
  \(\mathbf{d}''(0) = -6\mathbf{m} + 2\alpha(-\nabla f) + 2\mathbf{d}_{\text{L-BFGS}}\)
\end{itemize}

\textbf{Theorem B.7} (Cubic Convergence Properties): The cubic variant maintains all convergence guarantees of the quadratic version while potentially improving the convergence constant through momentum acceleration.

\hypertarget{b.5.3-trust-region-integration}{%
\subsubsection{B.5.3 Trust Region Integration}\label{b.5.3-trust-region-integration}}

QQN naturally extends to trust regions by constraining the univariate search:
\[t^* = \arg\min_{t: \|\mathbf{d}(t)\| \leq \Delta} f(\mathbf{x} + \mathbf{d}(t))\]

where \(\Delta\) is the trust region radius.

\textbf{Proposition B.2} (Trust Region Feasibility): For any \(\Delta > 0\), there exists \(t_{\max} > 0\) such that \(\|\mathbf{d}(t)\| \leq \Delta\) for all \(t \in [0, t_{\max}]\).

\emph{Proof}: Since \(\mathbf{d}(0) = \mathbf{0}\) and \(\mathbf{d}\) is continuous, by the intermediate value theorem, the set \(\{t : \|\mathbf{d}(t)\| \leq \Delta\}\) contains an interval \([0, t_{\max}]\) for some \(t_{\max} > 0\). \(\square\)

\hypertarget{b.6-comparison-with-related-methods}{%
\subsection{B.6 Comparison with Related Methods}\label{b.6-comparison-with-related-methods}}

\hypertarget{b.6.1-relationship-to-trust-region-methods}{%
\subsubsection{B.6.1 Relationship to Trust Region Methods}\label{b.6.1-relationship-to-trust-region-methods}}

Trust region methods solve:
\[\min_{\mathbf{s}} \mathbf{g}^T\mathbf{s} + \frac{1}{2}\mathbf{s}^T\mathbf{B}\mathbf{s} \quad \text{s.t.} \quad \|\mathbf{s}\| \leq \Delta\]

QQN can be viewed as solving a related but different problem:
\[\min_{t \geq 0} f(\mathbf{x} + \mathbf{d}(t))\]

where \(\mathbf{d}(t)\) is the quadratic path.
\textbf{Key differences}:

\begin{itemize}
\tightlist
\item
  Trust region: Solves 2D subproblem, then line search
\item
  QQN: Direct 1D optimization along quadratic path
\item
  Trust region: Requires trust region radius management
\item
  QQN: Parameter-free, automatic adaptation
\end{itemize}

\hypertarget{b.6.2-relationship-to-line-search-methods}{%
\subsubsection{B.6.2 Relationship to Line Search Methods}\label{b.6.2-relationship-to-line-search-methods}}

Traditional line search methods optimize:
\[\min_{\alpha > 0} f(\mathbf{x} + \alpha \mathbf{d})\]

QQN generalizes this by optimizing along a parametric path:
\[\min_{t \geq 0} f(\mathbf{x} + \mathbf{d}(t))\]

The key insight is that the direction itself changes with the parameter, providing additional flexibility.

\hypertarget{b.6.3-relationship-to-hybrid-methods}{%
\subsubsection{B.6.3 Relationship to Hybrid Methods}\label{b.6.3-relationship-to-hybrid-methods}}

Previous hybrid approaches typically use discrete switching:
\[\mathbf{d} = \begin{cases}
\mathbf{d}_{\text{gradient}} & \text{if condition A} \\
\mathbf{d}_{\text{quasi-Newton}} & \text{if condition B}
\end{cases}\]

QQN provides continuous interpolation, eliminating discontinuities and the need for switching logic.
