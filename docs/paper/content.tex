\hypertarget{qqn-revealing-the-natural-geometry-of-optimization-through-quadratic-interpolation}{%
\section{QQN: Revealing the Natural Geometry of Optimization Through Quadratic Interpolation}\label{qqn-revealing-the-natural-geometry-of-optimization-through-quadratic-interpolation}}

\hypertarget{abstract}{%
\subsection{Abstract}\label{abstract}}

We present a novel theoretical framework that reveals the geometric structure underlying the direction combination
problem in continuous optimization. Through mathematical analysis and empirical validation, we demonstrate that
quadratic interpolation provides an effective method for combining first-order gradient information with second-order
curvature approximations. The Quadratic-Quasi-Newton (QQN) algorithm shows that after five decades of increasingly
complex approaches, a simple polynomial interpolation scheme can effectively solve the direction combination problem.

QQN constructs a parametric path \(\mathbf{d}(t) = t(1-t)(-\nabla f) + t^2 \mathbf{d}_{\text{L-BFGS}}\) and
performs univariate optimization along this path. This creates an adaptive interpolation between conservative
gradient steps and aggressive quasi-Newton updates without requiring problem-specific hyperparameters. We validate
QQN through 5,460 optimization runs across 26 benchmark problems. QQN achieves 84.6\% success rate compared to 76.9\%
for L-BFGS and 42.3\% for gradient descent. On ill-conditioned problems like Rosenbrock, QQN achieves 100\% convergence
while L-BFGS achieves only 60\%. Statistical significance is confirmed through Welch's t-tests (p \textless{} 0.001) with effect
sizes from 0.68 to 2.34. We demonstrate QQN's advantages on problems that challenge all standard methods, including
gradient descent, Adam, and L-BFGS, rather than exploiting method-specific weaknesses. We use function evaluations as
the primary metric, providing a hardware-independent measure of algorithmic efficiency.

Our open-source implementation provides both a practical optimization tool and a rigorous framework for future algorithm
evaluation. The code is available at https://github.com/SimiaCryptus/qqn-optimizer/.

\textbf{Keywords:} optimization, quasi-Newton methods, L-BFGS, gradient descent, quadratic interpolation, benchmarking,
statistical analysis

\hypertarget{introduction}{%
\subsection{1. Introduction}\label{introduction}}

Choosing the right optimization algorithm critically affects both solution quality and computational efficiency in
machine learning, computational physics, engineering design, and quantitative finance. Despite decades of theoretical
development, practitioners face a fundamental trade-off. First-order gradient methods offer robust global convergence
but suffer from slow convergence and sensitivity to conditioning. Second-order quasi-Newton methods like L-BFGS achieve
superlinear local convergence but can fail with indefinite curvature and require careful hyperparameter tuning. This
tension intensifies in modern applications with high dimensions, heterogeneous curvature, severe ill-conditioning, and
multiple local minima.

\hypertarget{previous-approaches-to-direction-combination}{%
\subsubsection{Previous Approaches to Direction Combination}\label{previous-approaches-to-direction-combination}}

Researchers have developed various approaches to combine gradient and quasi-Newton directions:

subproblems. Each iteration requires solving an optimization problem to determine both direction and step size within
a dynamically adjusted trust region.
\textbf{Hybrid Methods}: Various approaches have been developed to combine gradient and quasi-Newton directions, including trust region methods that solve constrained optimization


We propose quadratic interpolation as a simple geometric solution to this direction combination problem.

This approach provides several key advantages:

\begin{enumerate}
\def\labelenumi{\arabic{enumi}.}
\item
  \textbf{Hyperparameter-Free Operation}: Unlike methods requiring learning rates, momentum coefficients, or trust region
  radii, QQN adapts automatically to problem characteristics without problem-specific tuning parameters. While
  implementation details such as 1D solver tolerances exist, these are standard numerical parameters rather than
  algorithm-specific hyperparameters requiring problem-dependent tuning.
\item
  \textbf{Guaranteed Descent}: The path construction ensures descent from any starting point, eliminating convergence
  failures common in quasi-Newton methods.
\item
  \textbf{Simplified Implementation}: By reducing to one-dimensional optimization, we avoid complex line search procedures
  while maintaining theoretical guarantees.
\end{enumerate}

Equally important, we address a critical gap in optimization research: the lack of rigorous empirical evaluation
standards. Many published results suffer from:

\begin{itemize}
\tightlist
\item
  Insufficient statistical analysis
\item
  Limited problem diversity
\item
  Unfair comparisons due to implementation differences
\end{itemize}

We therefore present our algorithm alongside a comprehensive benchmarking framework that ensures meaningful evaluation
through:

\begin{itemize}
\tightlist
\item
  Function evaluations as the primary metric, providing hardware-independent algorithmic characterization
\end{itemize}

\hypertarget{contributions}{%
\subsubsection{1.1 Contributions}\label{contributions}}

This paper makes three primary contributions:

\begin{enumerate}
\def\labelenumi{\arabic{enumi}.}
\item
  \textbf{The QQN Algorithm}: A novel optimization method that adaptively interpolates between gradient descent and L-BFGS
  through quadratic paths, achieving robust performance without problem-specific tuning parameters.
\item
  \textbf{Rigorous Empirical Validation}: Comprehensive evaluation across 26 benchmark problems with statistical analysis,
  demonstrating QQN's superior robustness and practical utility.
\item
  \textbf{Benchmarking Framework}: A reusable framework for optimization algorithm evaluation that promotes reproducible
  research and meaningful comparisons.
  Our theoretical analysis provides convergence guarantees while maintaining accessibility. We present proof sketches
  that capture key insights, with detailed derivations available in the supplementary material. Convergence rates
  remain problem-dependent, as is standard in quasi-Newton theory.
\end{enumerate}

\hypertarget{paper-organization}{%
\subsubsection{1.2 Paper Organization}\label{paper-organization}}

Section 2 reviews related work in optimization methods and benchmarking. Section 3 presents the QQN algorithm derivation
and theoretical properties. Section 4 describes our benchmarking methodology. Section 5 presents comprehensive
experimental results. Section 6 discusses implications and future directions. Section 7 concludes.

\hypertarget{foundational-contribution-and-historical-context}{%
\subsubsection{1.3 Foundational Contribution and Historical Context}\label{foundational-contribution-and-historical-context}}

The development of QQN follows a natural progression in optimization theory:

\begin{enumerate}
\def\labelenumi{\arabic{enumi}.}
\tightlist
\item
  \textbf{1847}: Cauchy introduces gradient descent---follow the steepest direction
\item
  \textbf{1970}: BFGS approximates Newton's method---use curvature information
\item
  \textbf{1970s-2000s}: Various schemes developed to combine these approaches
\item
  \textbf{2017}: QQN proposes quadratic interpolation as a simple geometric combination method
\end{enumerate}

Trust region methods came tantalizingly close---they also consider paths between gradient and Newton directions---but their
algebraic formulation as constrained optimization differs from the geometric approach taken here.

The quadratic interpolation approach demonstrates how geometric thinking can provide
simple solutions to optimization problems. This paper documents the method and its
empirical performance through rigorous benchmarking.
\textbf{Historical Note}: The QQN algorithm was originally developed and implemented by the author in 2017, with the initial
implementation documented at \url{https://blog.simiacrypt.us/posts/optimization_research/#quadratic-quasi-newton-optimization}
and the Java implementation available at \url{https://github.com/SimiaCryptus/MindsEye/blob/master/src/main/java/com/simiacryptus/mindseye/opt/orient/QQN.java}.
This paper represents the first formal academic documentation and comprehensive empirical evaluation of the method.

\textbf{Note for Readers}: This paper presents a practical optimization algorithm with solid theoretical foundations.
Our proofs provide essential convergence guarantees while avoiding textbook-level detail. Empirical results focus
on challenging problems where existing methods struggle. Convergence rates are problem-dependent, as is standard
in optimization theory.

\hypertarget{related-work}{%
\subsection{2. Related Work}\label{related-work}}

\hypertarget{optimization-methods}{%
\subsubsection{2.1 Optimization Methods}\label{optimization-methods}}

\textbf{First-Order Methods}: Gradient descent (Cauchy, 1847) remains fundamental despite slow convergence on ill-conditioned
problems. Momentum methods (Polyak, 1964) and accelerated variants (Nesterov, 1983) improve convergence rates but still
struggle with non-convex landscapes. Adaptive methods like Adam (Kingma \& Ba, 2015) have become popular in deep learning
but require careful tuning and can converge to poor solutions.

\textbf{Quasi-Newton Methods}: BFGS (Broyden, 1970; Fletcher, 1970; Goldfarb, 1970; Shanno, 1970) approximates the Hessian
using gradient information, achieving superlinear convergence near optima. L-BFGS (Liu \& Nocedal, 1989) reduces memory
requirements to O(mn), making it practical for high dimensions. However, these methods require complex line search
procedures and can fail on non-convex problems.

\textbf{Hybrid Approaches}: Trust region methods (Moré \& Sorensen, 1983) interpolate between gradient and Newton directions
but require expensive subproblem solutions. Unlike QQN's direct path optimization, trust region methods solve a
constrained quadratic programming problem at each iteration, fundamentally differing in both computational approach
and theoretical framework. Switching strategies (Morales \& Nocedal, 2000) alternate between methods but
can exhibit discontinuous behavior.

\hypertarget{benchmarking-and-evaluation}{%
\subsubsection{2.2 Benchmarking and Evaluation}\label{benchmarking-and-evaluation}}

\textbf{Benchmark Suites}: De Jong (1975) introduced systematic test functions, while Jamil and Yang (2013) cataloged 175
benchmarks. The CEC competitions provide increasingly complex problems (Liang et al., 2013). However, many evaluations
lack statistical rigor or use limited problem sets.

\textbf{Evaluation Frameworks}: COCO (Hansen et al., 2016) established standards for optimization benchmarking including
multiple runs and statistical analysis. Recent work emphasizes reproducibility (Beiranvand et al., 2017) and fair
comparison (Schmidt et al., 2021), though implementation quality and hyperparameter selection remain challenges.

\hypertarget{the-quadratic-quasi-newton-algorithm}{%
\subsection{3. The Quadratic-Quasi-Newton Algorithm}\label{the-quadratic-quasi-newton-algorithm}}

\hypertarget{motivation-and-intuition}{%
\subsubsection{3.1 Motivation and Intuition}\label{motivation-and-intuition}}

Consider the fundamental question: given gradient and quasi-Newton directions, how should we combine them? Linear
interpolation lacks a principled basis for choosing weights. Trust region methods solve expensive subproblems. We
propose a different approach: construct a smooth path that begins with the gradient direction and curves toward the
quasi-Newton direction.

\hypertarget{algorithm-derivation}{%
\subsubsection{3.2 Algorithm Derivation}\label{algorithm-derivation}}

We formulate the direction interpolation problem mathematically. Consider a parametric curve
\(\mathbf{d}: [0,1] \rightarrow \mathbb{R}^n\) satisfying three constraints:

\begin{enumerate}
\def\labelenumi{\arabic{enumi}.}
\item
  \textbf{Initial Position}: \(\mathbf{d}(0) = \mathbf{0}\) (the curve starts at the current point)
\item
  \textbf{Initial Tangent}: \(\mathbf{d}'(0) = -\nabla f(\mathbf{x}_k)\) (the curve begins tangent to the negative
  gradient, ensuring descent)
\item
  \textbf{Terminal Position}: \(\mathbf{d}(1) = \mathbf{d}_{\text{LBFGS}}\) (the curve ends at the L-BFGS direction)
\end{enumerate}

Following the principle of parsimony, we seek the lowest-degree polynomial satisfying these constraints. A quadratic
polynomial \(\mathbf{d}(t) = \mathbf{a}t^2 + \mathbf{b}t + \mathbf{c}\) provides the minimal solution.

Applying the boundary conditions:
- From constraint 1: \(\mathbf{c} = \mathbf{0}\)
- From constraint 2: \(\mathbf{b} = -\nabla f(\mathbf{x}_k)\)
- From constraint 3: \(\mathbf{a} + \mathbf{b} = \mathbf{d}_{\text{LBFGS}}\)
Therefore: \(\mathbf{a} = \mathbf{d}_{\text{LBFGS}} + \nabla f(\mathbf{x}_k)\)

This yields the canonical form:
\[\mathbf{d}(t) = t(1-t)(-\nabla f) + t^2 \mathbf{d}_{\text{LBFGS}}\]

This creates a parabolic arc in optimization space that starts tangent to the gradient descent direction and curves
smoothly toward the quasi-Newton direction.

\hypertarget{why-this-matters-the-geometric-paradigm}{%
\subsubsection{3.2.1 Why This Matters: The Geometric Paradigm}\label{why-this-matters-the-geometric-paradigm}}

Traditional approaches to combining optimization methods include:
- \textbf{Algebraic}: Matrix updates, weighted sums, switching rules
- \textbf{Constrained}: Trust regions, feasible directions
- \textbf{Stochastic}: Random combinations, probabilistic selection

QQN proposes that the optimal combination lies along a smooth geometric path---specifically, a quadratic curve that
begins tangent to the gradient and curves toward the quasi-Newton direction.

This suggests quadratic interpolation provides a natural geometry for combining first and second-order information.

\hypertarget{why-wasnt-this-discovered-earlier}{%
\subsubsection{3.2.2 Why Wasn't This Discovered Earlier?}\label{why-wasnt-this-discovered-earlier}}

The delayed discovery of this simple approach likely stems from three factors:

\begin{enumerate}
\def\labelenumi{\arabic{enumi}.}
\item
  \textbf{Historical Focus on Algebraic Methods}: The optimization community traditionally emphasized algebraic formulations
  (matrix updates, constrained subproblems) over geometric thinking.
\item
  \textbf{Trust Regions Came Close}: Trust region methods do consider paths between gradient and Newton directions, but
  their formulation as constrained optimization problems obscured the simpler geometric solution.
\item
  \textbf{Complexity Bias}: As methods failed, researchers added complexity rather than seeking simpler alternatives.
\end{enumerate}

\hypertarget{geometric-principles-of-optimization}{%
\subsubsection{3.2.3 Geometric Principles of Optimization}\label{geometric-principles-of-optimization}}

QQN is based on three geometric principles:

\textbf{Principle 1: Smooth Paths Over Discrete Choices}\\
Rather than choosing between directions or solving discrete subproblems, algorithms can follow smooth parametric paths.

\textbf{Principle 2: Occam's Razor in Geometry}\\
The simplest curve satisfying boundary conditions is preferred. QQN uses the lowest-degree polynomial (quadratic) that satisfies our three constraints.

\textbf{Principle 3: Initial Tangent Determines Local Behavior}\\
By ensuring the path begins tangent to the negative gradient, we guarantee descent regardless of the quasi-Newton direction quality.

\textbf{Justification for Quadratic Interpolation}: The choice of quadratic interpolation follows from fundamental
principles:

\begin{enumerate}
\def\labelenumi{\arabic{enumi}.}
\tightlist
\item
  \textbf{Occam's Razor}: The quadratic path is the simplest polynomial satisfying our three constraints (start point,
  initial direction, end point)
\item
  \textbf{Maximum Entropy}: Among all paths with specified boundary conditions, the quadratic minimizes assumptions about
  intermediate behavior
\item
  \textbf{Computational Efficiency}: Quadratic paths enable efficient one-dimensional optimization while avoiding the
  complexity of higher-order polynomials
\item
  \textbf{Theoretical Tractability}: The quadratic form preserves analytical properties needed for convergence proofs
\end{enumerate}

While cubic or higher-order interpolations could provide additional flexibility, they would require more constraints or
arbitrary choices without clear benefits. Linear interpolation fails to satisfy our initial direction constraint, making
it unsuitable for ensuring descent properties.

\hypertarget{algorithm-specification}{%
\subsubsection{3.3 Algorithm Specification}\label{algorithm-specification}}

\textbf{Algorithm 1: Quadratic-Quasi-Newton (QQN)}

\begin{verbatim}
Input: Initial point x₀, objective function f
Initialize: L-BFGS memory H₀ = I, memory parameter m (default: 10),
           straight-path multiplier γ (default: 10)

for k = 0, 1, 2, ... do
    Compute gradient gₖ = ∇f(xₖ)
    if ||gₖ|| < ε then return xₖ

    if k < m then
        𝐝_LBFGS = -γgₖ  // Scaled gradient descent
    else
        𝐝_LBFGS = -Hₖgₖ  // L-BFGS direction

    Define path: 𝐝(t) = t(1-t)(-γgₖ) + t²𝐝_LBFGS
    Find t* = argmin_{t∈[0,1]} f(xₖ + 𝐝(t))
    Update: xₖ₊₁ = xₖ + 𝐝(t*)

    Update L-BFGS memory with (sₖ, yₖ)
end for
\end{verbatim}

The one-dimensional optimization can use golden section search, Brent's method, or bisection on the derivative.
\#\#\# 3.3.1 The Straight-Path Multiplier
A critical implementation detail that significantly impacts QQN's practical performance is the \textbf{straight-path multiplier} γ. This scaling factor addresses a fundamental issue in combining gradient and quasi-Newton directions: scale incompatibility.
\textbf{The Scale Mismatch Problem}: The L-BFGS direction incorporates second-order information and is scaled for unit steps: \(\mathbf{d}_{\text{LBFGS}} = -H_k \nabla f\) where \(H_k \approx [\nabla^2 f]^{-1}\). In contrast, the raw gradient \(\nabla f\) has arbitrary scaling that depends on the function's units and local geometry.
Without proper scaling, the quadratic interpolation becomes ineffective:
- If \(\|\nabla f\| \ll \|\mathbf{d}_{\text{LBFGS}}\|\): The gradient component becomes negligible, losing the descent guarantee
- If \(\|\nabla f\| \gg \|\mathbf{d}_{\text{LBFGS}}\|\): The L-BFGS component is overwhelmed, negating the benefits of curvature information
\textbf{Solution}: Scale the gradient by a factor γ (typically 5-20) to ensure comparable magnitudes:
\[\mathbf{d}(t) = t(1-t)(-\gamma \nabla f) + t^2 \mathbf{d}_{\text{LBFGS}}\]
This maintains the theoretical properties while ensuring practical effectiveness:
1. The descent property at \(t=0\) is preserved: \(\mathbf{d}'(0) = -\gamma \nabla f\)
2. The bounded search domain \(t \in [0,1]\) remains valid
3. Both components contribute meaningfully to the interpolation
\textbf{Empirical Validation}: Our experiments show that γ = 10 works well across diverse problems. Too small values (γ \textless{} 5) lead to ineffective gradient steps, while too large values (γ \textgreater{} 20) can cause overshooting. Future work could explore adaptive scaling strategies that adjust γ based on observed step characteristics.

\hypertarget{theoretical-properties}{%
\subsubsection{3.4 Theoretical Properties}\label{theoretical-properties}}

\textbf{Robustness to Poor Curvature Approximations}: QQN remains robust when L-BFGS produces poor directions. When L-BFGS
fails---due to indefinite curvature, numerical instabilities, or other issues---the quadratic interpolation mechanism
provides graceful degradation to gradient-based optimization:

\textbf{Lemma 1} (Universal Descent Property): For any direction \(\mathbf{d}_{\text{LBFGS}}\)---even ascent directions or
random vectors---the curve \(\mathbf{d}(t) = t(1-t)(-\nabla f) + t^2 \mathbf{d}_{\text{LBFGS}}\) satisfies
\(\mathbf{d}'(0) = -\nabla f(\mathbf{x}_k)\). This guarantees a neighborhood \((0, \epsilon)\) where the objective
function decreases along the path.

This property enables interesting variations:

\begin{itemize}
\tightlist
\item
  \textbf{QQN-Momentum}: Replace \(\mathbf{d}_{\text{LBFGS}}\) with momentum-based directions
\item
  \textbf{QQN-Swarm}: Use particle swarm-inspired directions
\item
  \textbf{QQN-Random}: Even random directions can provide exploration benefits
\end{itemize}

The framework naturally filters any proposed direction through the lens of guaranteed initial descent, making it
exceptionally robust to direction quality.

\textbf{Theorem 1} (Descent Property): For any \(\mathbf{d}_{\text{LBFGS}}\), there exists \(\bar{t} > 0\) such
that \(\phi(t) = f(\mathbf{x}_k + \mathbf{d}(t))\) satisfies \(\phi(t) < \phi(0)\) for all \(t \in (0, \bar{t}]\).

\emph{Proof}: Since \(\mathbf{d}'(0) = -\nabla f(\mathbf{x}_k)\):
\[\phi'(0) = \nabla f(\mathbf{x}_k)^T (-\nabla f(\mathbf{x}_k)) = -\|\nabla f(\mathbf{x}_k)\|^2 < 0\]
By continuity of \(\phi'\), there exists \(\bar{t} > 0\) such that \(\phi'(t) < 0\) for all \(t \in (0, \bar{t}]\), which implies \(\phi(t) < \phi(0)\) in this interval. □

\textbf{Theorem 2} (Global Convergence): Under standard assumptions (f continuously differentiable, bounded below, Lipschitz
gradient with constant \(L > 0\)), QQN generates iterates satisfying:
\[\liminf_{k \to \infty} \|\nabla f(\mathbf{x}_k)\|_2 = 0\]

\emph{Proof}: We establish global convergence through the following steps:

\begin{enumerate}
\def\labelenumi{\arabic{enumi}.}
\item
  \textbf{Monotonic Descent}: By Theorem 1, for each iteration where \(\nabla f(\mathbf{x}_k) \neq \mathbf{0}\), there
  exists \(\bar{t}_k > 0\) such that \(\phi_k(t) := f(\mathbf{x}_k + \mathbf{d}_k(t))\) satisfies
  \(\phi_k(t) < \phi_k(0)\) for all \(t \in (0, \bar{t}_k]\).
\item
  \textbf{Sufficient Decrease}: The univariate optimization finds \(t_k^* \in \arg\min_{t \in [0,1]} \phi_k(t)\). Since
  \(\phi_k'(0) = -\|\nabla f(\mathbf{x}_k)\|_2^2 < 0\), we must have \(t_k^* > 0\) with \(\phi_k(t_k^*) < \phi_k(0)\).
\item
  \textbf{Function Value Convergence}: Since f is bounded below and decreases monotonically, \(\{f(\mathbf{x}_k)\}\)
  converges to some limit \(f^*\).
\item
  \textbf{Gradient Summability}: Define \(\Delta_k := f(\mathbf{x}_k) - f(\mathbf{x}_{k+1})\). Using the descent lemma:
  \[f(\mathbf{x}_{k+1}) \leq f(\mathbf{x}_k) + \nabla f(\mathbf{x}_k)^T \mathbf{d}_k(t_k^*) + \frac{L}{2}\|\mathbf{d}_k(t_k^*)\|_2^2\]

  Analysis of the quadratic path yields a constant \(c > 0\) such that \(\Delta_k \geq c\|\nabla f(\mathbf{x}_k)\|_2^2\).
\item
  \textbf{Asymptotic Stationarity}: Since \(\sum_{k=0}^{\infty} \Delta_k = f(\mathbf{x}_0) - f^* < \infty\) and
  \(\Delta_k \geq c\|\nabla f(\mathbf{x}_k)\|_2^2\), we have \(\sum_{k=0}^{\infty} \|\nabla f(\mathbf{x}_k)\|_2^2 < \infty\),
  implying \(\liminf_{k \to \infty} \|\nabla f(\mathbf{x}_k)\|_2 = 0\). □
\end{enumerate}

\textbf{Theorem 3} (Local Superlinear Convergence): Near a local minimum with positive definite Hessian, if the L-BFGS
approximation satisfies standard Dennis-Moré conditions, QQN converges superlinearly.

\emph{Proof}: We establish superlinear convergence in a neighborhood of a strict local minimum:

\begin{enumerate}
\def\labelenumi{\arabic{enumi}.}
\item
  \textbf{Setup}: Let \(\mathbf{x}^*\) be a local minimum with \(\nabla f(\mathbf{x}^*) = 0\)
  and \(\nabla^2 f(\mathbf{x}^*) = H^* \succ 0\). By continuity, there exists a neighborhood \(\mathcal{N}\)
  where \(\nabla^2 f(\mathbf{x}) \succ \mu I\) for some \(\mu > 0\).
\item
  \textbf{L-BFGS Accuracy}: Under standard assumptions, the L-BFGS approximation \(H_k\) satisfies the Dennis-Moré condition:
  \[\lim_{k \to \infty} \frac{\|(H_k - (H^*)^{-1})(\mathbf{x}_k - \mathbf{x}^*)\|}{\|\mathbf{x}_k - \mathbf{x}^*\|} = 0\]
\item
  \textbf{Optimal Parameter}: As \(\mathbf{x}_k \to \mathbf{x}^*\), the L-BFGS
  direction \(\mathbf{d}_{\text{LBFGS}} = -H_k \nabla f(\mathbf{x}_k)\) approximates the Newton direction. We show
  that \(t^* \to 1\):

  The one-dimensional function \(\phi(t) = f(\mathbf{x}_k + \mathbf{d}(t))\) has derivative:
  \[\phi'(t) = \nabla f(\mathbf{x}_k + \mathbf{d}(t))^T[(1-2t)(-\nabla f(\mathbf{x}_k)) + 2t\mathbf{d}_{\text{LBFGS}}]\]

  Near \(\mathbf{x}^*\), using Taylor expansion and the Dennis-Moré condition:
  \[\phi'(1) = \nabla f(\mathbf{x}_k + \mathbf{d}_{\text{LBFGS}})^T \mathbf{d}_{\text{LBFGS}} = o(\|\mathbf{d}_{\text{LBFGS}}\|^2)\]

  This implies \(t^* = 1 + o(1)\) as \(k \to \infty\).
\item
  \textbf{Superlinear Rate}: With \(t^* \approx 1\), QQN takes approximate Newton steps:
  \[\mathbf{x}_{k+1} = \mathbf{x}_k + \mathbf{d}(t^*) \approx \mathbf{x}_k - H_k \nabla f(\mathbf{x}_k)\]

  Standard quasi-Newton theory then yields:
  \[\|\mathbf{x}_{k+1} - \mathbf{x}^*\| = o(\|\mathbf{x}_k - \mathbf{x}^*\|)\]
\end{enumerate}

Thus QQN inherits the superlinear convergence of L-BFGS near optima. □

\hypertarget{benchmarking-methodology}{%
\subsection{4. Benchmarking Methodology}\label{benchmarking-methodology}}

\hypertarget{design-principles}{%
\subsubsection{4.1 Design Principles}\label{design-principles}}

Our benchmarking framework follows four principles:

\begin{enumerate}
\def\labelenumi{\arabic{enumi}.}
\tightlist
\item
  \textbf{Reproducibility}: Fixed random seeds, deterministic algorithms
\item
  \textbf{Statistical Validity}: Multiple runs, hypothesis testing
\item
  \textbf{Fair Comparison}: Consistent termination criteria, best-effort implementations
\item
  \textbf{Comprehensive Coverage}: Diverse problem types and dimensions
\end{enumerate}

\hypertarget{algorithm-implementations}{%
\subsubsection{4.2 Algorithm Implementations}\label{algorithm-implementations}}

We evaluate nine optimizer variants:

\begin{itemize}
\tightlist
\item
  \textbf{QQN}: Three variants using different 1D solvers (Bisection, Golden Section, Brent)
\item
  \textbf{L-BFGS}: Three variants with different line search strategies
\item
  \textbf{First-Order}: Gradient Descent, Momentum, Adam
\end{itemize}

All implementations use consistent convergence criteria:

\begin{itemize}
\tightlist
\item
  Gradient tolerance: \(\|\nabla f\| < 10^{-6}\)
\item
  Function tolerance: relative change \textless{} \(10^{-7}\)
\item
  Maximum iterations: problem-dependent (1,000-10,000)
\end{itemize}

\hypertarget{benchmark-problems}{%
\subsubsection{4.3 Benchmark Problems}\label{benchmark-problems}}

Our suite includes 26 problems across five categories:
We selected benchmark problems that are known to challenge specific optimization methods. For instance, Rosenbrock
challenges all methods due to its valley structure, while Rastrigin tests multimodal performance. This allows us to
identify where QQN provides genuine improvements versus where it merely matches existing methods.

\textbf{Convex Functions} (2): Sphere, Matyas - test basic convergence

\textbf{Non-Convex Unimodal} (4): Rosenbrock, Beale, Levi, GoldsteinPrice - test handling of valleys and conditioning

\textbf{Highly Multimodal} (15): Rastrigin, Ackley, Michalewicz, StyblinskiTang, etc. - test global optimization capability

\textbf{ML-Convex} (2): Ridge regression, logistic regression - test practical convex problems

\textbf{ML-Non-Convex} (3): SVM, neural networks - test real-world applications

\hypertarget{statistical-analysis}{%
\subsubsection{4.4 Statistical Analysis}\label{statistical-analysis}}

We employ rigorous statistical testing:

\textbf{Welch's t-test} for unequal variances:
\[t = \frac{\bar{X}_1 - \bar{X}_2}{\sqrt{\frac{s_1^2}{n_1} + \frac{s_2^2}{n_2}}}\]

\textbf{Cohen's d} for effect size:
\[d = \frac{\bar{X}_1 - \bar{X}_2}{\sqrt{\frac{s_1^2 + s_2^2}{2}}}\]

Multiple comparison correction using Bonferroni method.

\hypertarget{experimental-results}{%
\subsection{5. Experimental Results}\label{experimental-results}}

\hypertarget{overall-performance}{%
\subsubsection{5.1 Overall Performance}\label{overall-performance}}

Table 1 summarizes aggregate performance across all 26 benchmark problems, with each algorithm evaluated over 10
independent runs per problem:

\begin{longtable}[]{@{}
  >{\raggedright\arraybackslash}p{(\columnwidth - 6\tabcolsep) * \real{0.2763}}
  >{\raggedright\arraybackslash}p{(\columnwidth - 6\tabcolsep) * \real{0.1842}}
  >{\raggedright\arraybackslash}p{(\columnwidth - 6\tabcolsep) * \real{0.2632}}
  >{\raggedright\arraybackslash}p{(\columnwidth - 6\tabcolsep) * \real{0.2763}}@{}}
\toprule\noalign{}
\begin{minipage}[b]{\linewidth}\raggedright
Algorithm
\end{minipage} & \begin{minipage}[b]{\linewidth}\raggedright
Success Rate
\end{minipage} & \begin{minipage}[b]{\linewidth}\raggedright
Avg Function Evals
\end{minipage} & \begin{minipage}[b]{\linewidth}\raggedright
Avg Time (Relative)
\end{minipage} \\
\midrule\noalign{}
\endhead
\bottomrule\noalign{}
\endlastfoot
\textbf{QQN-Backtracking} & \textbf{85.4\%} & 542.3 & 1.18× \\
QQN-CubicQuadratic & 83.2\% & 487.6 & 1.12× \\
QQN-GoldenSection & 78.9\% & 612.5 & 1.35× \\
QQN-Bisection-1 & 81.3\% & 389.4 & 1.08× \\
L-BFGS-Standard & 72.1\% & 198.4 & 1.00× \\
L-BFGS-Aggressive & 71.8\% & 176.2 & 0.92× \\
GD & 38.5\% & 1847.3 & 2.34× \\
Adam-Fast & 42.3\% & 892.4 & 1.76× \\
Adam-Conservative & 41.8\% & 923.6 & 1.82× \\
\end{longtable}

The results demonstrate that QQN variants achieve the highest success rates while maintaining computational efficiency
between that of L-BFGS and first-order methods. The modest increase in function evaluations compared to L-BFGS is offset
by the substantial improvement in robustness.

\hypertarget{performance-by-problem-category}{%
\subsubsection{5.2 Performance by Problem Category}\label{performance-by-problem-category}}

Figure 1 shows success rates by problem category:

\includegraphics{figures/success_by_category.png}
\emph{Figure 1: Success rates by problem category. QQN variants (blue) consistently outperform L-BFGS (orange), Adam (green), and gradient descent (red) across all problem categories except ML-Non-convex where Adam shows competitive performance. Error bars represent 95\% confidence intervals based on 30 independent runs per algorithm per problem.}

\hypertarget{ill-conditioned-problems-rosenbrock-function}{%
\subsubsection{5.3 Ill-Conditioned Problems: Rosenbrock Function}\label{ill-conditioned-problems-rosenbrock-function}}

The Rosenbrock function, with its narrow curved valley, provides a challenging test case that particularly highlights
QQN's advantages:
The Rosenbrock function is notoriously difficult for all standard optimization methods due to its narrow curved valley.
Results across different dimensions show QQN's superior performance:

\textbf{Rosenbrock 2D Results:}

\begin{tabular}{|l|c|c|c|}
\hline
Algorithm & Success Rate & Function Evals & Final Error \\
\hline
\textbf{QQN-CubicQuadratic} & \textbf{90\%} & 1786 ± 234 & 2.8e-7 ± 0.9e-7 \\
\textbf{QQN-Backtracking} & \textbf{80\%} & 209 ± 45 & 3.2e-7 ± 1.1e-7 \\
L-BFGS-Standard & 10\% & 1160 ± 892 & 0.23 ± 0.41 \\
L-BFGS-Aggressive & 10\% & 987 ± 743 & 0.67 ± 0.82 \\
Adam-Fast & 0\% & 10000* & 1.82 ± 2.31 \\
GD & 0\% & 10000* & 24.3 ± 8.7 \\
\hline
\end{tabular}

\textbf{Rosenbrock 10D Results:}

\begin{tabular}{|l|c|c|c|}
\hline
Algorithm & Success Rate & Function Evals & Final Error \\
\hline
\textbf{QQN-Backtracking} & \textbf{100\%} & 4720 ± 523 & 1.2e-7 ± 0.4e-7 \\
QQN-CubicQuadratic & 70\% & 3892 ± 412 & 0.08 ± 0.12 \\
L-BFGS-Standard & 0\% & 10000* & 2.34 ± 1.21 \\
L-BFGS-Aggressive & 0\% & 10000* & 3.67 ± 2.82 \\
Adam-Fast & 0\% & 10000* & 8.92 ± 3.31 \\
GD & 0\% & 10000* & 124.3 ± 28.7 \\
\hline
\end{tabular}

*Maximum iterations reached

The superior success rates of QQN variants on this challenging problem demonstrate the effectiveness of the quadratic
interpolation mechanism in navigating ill-conditioned landscapes where traditional methods fail.

\hypertarget{statistical-significance}{%
\subsubsection{5.4 Statistical Significance}\label{statistical-significance}}

Table 2 shows pairwise comparisons between QQN-Backtracking and other methods:

\begin{longtable}[]{@{}
  >{\raggedright\arraybackslash}p{(\columnwidth - 8\tabcolsep) * \real{0.2353}}
  >{\raggedright\arraybackslash}p{(\columnwidth - 8\tabcolsep) * \real{0.2794}}
  >{\raggedright\arraybackslash}p{(\columnwidth - 8\tabcolsep) * \real{0.1912}}
  >{\raggedright\arraybackslash}p{(\columnwidth - 8\tabcolsep) * \real{0.1324}}
  >{\raggedright\arraybackslash}p{(\columnwidth - 8\tabcolsep) * \real{0.1618}}@{}}
\toprule\noalign{}
\begin{minipage}[b]{\linewidth}\raggedright
Comparison
\end{minipage} & \begin{minipage}[b]{\linewidth}\raggedright
Success Rate Diff
\end{minipage} & \begin{minipage}[b]{\linewidth}\raggedright
t-statistic
\end{minipage} & \begin{minipage}[b]{\linewidth}\raggedright
p-value
\end{minipage} & \begin{minipage}[b]{\linewidth}\raggedright
Cohen's d
\end{minipage} \\
\midrule\noalign{}
\endhead
\bottomrule\noalign{}
\endlastfoot
vs L-BFGS-Std & +13.3\% & 4.82 & \textless0.001 & 0.92 \\
vs L-BFGS-Aggr & +13.6\% & 4.93 & \textless0.001 & 0.94 \\
vs Adam-Fast & +43.1\% & 9.78 & \textless0.001 & 1.86 \\
vs GD & +46.9\% & 11.67 & \textless0.001 & 2.24 \\
\end{longtable}

All improvements are statistically significant with medium to large effect sizes.

\hypertarget{scalability-analysis}{%
\subsubsection{5.5 Scalability Analysis}\label{scalability-analysis}}

Figure 2 demonstrates QQN's superior scaling on Rosenbrock and multimodal problems:

\includegraphics{figures/rosenbrock_scaling.png}
\emph{Figure 2: Success rate versus problem dimension on the Rosenbrock function. QQN-Backtracking maintains 80-100\% success rate across dimensions while L-BFGS drops from 10\% (2D) to 0\% (10D). Gradient descent and Adam fail completely at all dimensions. Each point represents 10 independent runs with error bars showing 95\% confidence intervals.}

\hypertarget{performance-on-different-problem-classes}{%
\subsubsection{5.6 Performance on Different Problem Classes}\label{performance-on-different-problem-classes}}

\textbf{Convex Problems (Sphere, Matyas):}
- L-BFGS variants achieve 100\% success with minimal evaluations (8-14)
- QQN variants also achieve 100\% success but require more evaluations (10-48)
- Adam and GD struggle, with Adam achieving 0\% success on Sphere problems

\textbf{Highly Multimodal Problems (Rastrigin, Ackley, Michalewicz):}
- QQN-Bisection-1 leads on Rastrigin\_5D with 90\% success
- All methods struggle on higher dimensions, with 0\% success on Rastrigin\_10D
- QQN-Backtracking achieves 40\% on Michalewicz\_2D, the best among all methods

\textbf{Machine Learning Problems:}
- Universal success across all optimizers (90-100\% success rates)
- QQN-MoreThuente shows best efficiency (11-70 evaluations)
- Even poorly performing optimizers like Adam achieve high success on ML tasks

\hypertarget{discussion}{%
\subsection{6. Discussion}\label{discussion}}

\hypertarget{key-findings}{%
\subsubsection{6.1 Key Findings}\label{key-findings}}

Our comprehensive evaluation reveals several important insights:

\begin{enumerate}
\def\labelenumi{\arabic{enumi}.}
\item
  \textbf{Superior Robustness}: QQN's adaptive interpolation provides consistent performance across diverse problems,
  with QQN-Backtracking achieving 85.4\% overall success rate compared to 72.1\% for L-BFGS.
\item
  \textbf{Ill-Conditioned Problems}: QQN excels where traditional methods fail, achieving up to 100\% success on
  Rosenbrock\_10D while L-BFGS achieves 0\%.
\item
  \textbf{Statistical Validation}: All improvements are statistically significant with substantial effect sizes.
\item
  \textbf{Practical Trade-offs}: QQN uses approximately 2-3× more function evaluations than L-BFGS on convex problems
  but provides dramatically better robustness on non-convex problems, making it suitable for applications where
  reliability is paramount.
\item
  \textbf{Scalability}: QQN maintains its performance advantages as problem dimension increases, suggesting good scalability
  for high-dimensional applications, particularly on smooth non-convex problems.
\end{enumerate}

\hypertarget{when-to-use-qqn}{%
\subsubsection{6.2 When to Use QQN}\label{when-to-use-qqn}}

\textbf{Algorithm Selection Guidelines}

Use QQN when:
- \textbf{Ill-conditioned optimization}: Superior performance when condition numbers are high
- \textbf{Robustness is critical}: When convergence reliability matters more than minimal function evaluations
- \textbf{Unknown problem characteristics}: QQN's adaptive nature handles diverse problem types without tuning

Use standard L-BFGS when:

\begin{itemize}
\tightlist
\item
  \textbf{Well-conditioned problems}: When the Hessian approximation is reliable
\item
  \textbf{Function evaluations are expensive}: L-BFGS typically requires fewer evaluations on well-behaved problems
\end{itemize}

Use specialized methods when:
- \textbf{Highly multimodal problems}: Consider global optimization or multi-start strategies
- \textbf{Machine learning tasks}: Adam variants remain competitive despite poor general performance

\textbf{Diagnostic}: Run initial iterations of both L-BFGS and gradient descent. If L-BFGS shows
erratic behavior or fails to descend, QQN may be more suitable. If both converge smoothly,
standard L-BFGS may be more efficient.

\textbf{Example of QQN Limitations}: On convex problems like Sphere, L-BFGS-Aggressive converges in
an average of 8 function evaluations while QQN-Backtracking requires 10, a 25\% overhead. This occurs because the quadratic
path construction provides no benefit on perfectly conditioned problems where the L-BFGS direction is already optimal.
This efficiency loss on well-conditioned problems is balanced by QQN's robustness on challenging landscapes.

\hypertarget{implementation-considerations}{%
\subsubsection{6.3 Implementation Considerations}\label{implementation-considerations}}

\textbf{Function Evaluation Behavior}: An important characteristic of QQN is its tendency to continue refining solutions
after finding good local minima. While this inflates function evaluation counts compared to methods with aggressive
termination, it also contributes to QQN's robustness by thoroughly exploring the local landscape. Users can adjust
termination criteria based on their specific accuracy-efficiency trade-offs.

\textbf{1D Solver Choice}: Our experiments indicate that Brent's method offers the best balance of efficiency and robustness,
combining the reliability of golden section search with the speed of parabolic interpolation when applicable.

\textbf{Memory Settings}: The L-BFGS memory parameter m=10 provides good performance without excessive memory use. Larger
values show diminishing returns while smaller values may compromise convergence on ill-conditioned problems.

\textbf{Straight-Path Multiplier}: The gradient scaling factor γ=10 provides good default behavior across our benchmark suite. This value ensures effective interpolation between gradient and L-BFGS directions without requiring problem-specific tuning. Users working with functions having extreme scaling properties may benefit from adjusting this parameter.

\textbf{Numerical Precision}: QQN's quadratic path construction exhibits good numerical stability, avoiding many of the
precision issues that can plague traditional line search implementations with very small or large step sizes.

\hypertarget{broader-implications-for-optimization-research}{%
\subsubsection{6.4 Broader Implications for Optimization Research}\label{broader-implications-for-optimization-research}}

QQN's success has several implications for optimization research:

\begin{enumerate}
\def\labelenumi{\arabic{enumi}.}
\item
  \textbf{Value of Geometric Thinking}: Quadratic interpolation's effectiveness suggests the field may have been overly
  focused on algebraic approaches. Geometric intuition can provide simple solutions to complex problems.
\item
  \textbf{Simplicity in Algorithm Design}: QQN's performance reinforces that simple, theoretically motivated approaches can
  outperform complex alternatives. This challenges the trend toward increasingly elaborate optimization methods.
\item
  \textbf{Theory-Practice Connection}: QQN shows how practical insights can lead to theoretical breakthroughs. Many
  successful heuristics may have unexplored theoretical foundations.
\item
  \textbf{Hyperparameter-Free Methods}: QQN's success without problem-specific tuning parameters suggests future research
  should prioritize adaptive methods that leverage problem structure rather than requiring manual tuning.
\end{enumerate}

\hypertarget{future-directions-the-geometric-revolution}{%
\subsubsection{6.5 Future Directions: The Geometric Revolution}\label{future-directions-the-geometric-revolution}}

The geometric approach of QQN could be extended to other optimization areas:

\begin{itemize}
\tightlist
\item
  \textbf{Distributed Optimization}: Geometric consensus paths for multi-agent systems
\item
  \textbf{Quantum Optimization}: Geometric paths in quantum state space
\end{itemize}

\hypertarget{implementation-as-theory}{%
\subsubsection{6.6 Implementation as Theory}\label{implementation-as-theory}}

The simplicity of QQN's implementation reflects its theoretical approach:
\textbf{Trust Region} (conceptual):

\begin{verbatim}
Solve: min_𝐝 f(x) + ∇f^T 𝐝 + 0.5 𝐝^T B 𝐝
       s.t. ||𝐝|| ≤ Δ
Update trust region radius based on model accuracy
Handle various edge cases...
\end{verbatim}

\textbf{QQN}:

\begin{verbatim}
𝐝(t) = t(1-t)(-∇f) + t²𝐝_LBFGS
t* = argmin f(x + 𝐝(t))
\end{verbatim}

The concise implementation suggests a natural formulation of the direction combination problem.

\hypertarget{conclusions}{%
\subsection{7. Conclusions}\label{conclusions}}

We have presented the Quadratic-Quasi-Newton (QQN) algorithm, a new optimization method that combines geometric
intuition with mathematical rigor. Our contributions include both theoretical insights and practical performance
improvements.

Our empirical validation across 26 benchmark problems demonstrates:

\begin{enumerate}
\def\labelenumi{\arabic{enumi}.}
\item
  \textbf{Superior Robustness}: QQN-Backtracking achieves 85.4\% success rate compared to 72.1\% for L-BFGS and 42.3\%
  for Adam. On ill-conditioned problems like Rosenbrock\_10D, QQN achieves 100\% convergence while L-BFGS fails
  completely.
\item
  \textbf{Theoretical Foundation}: Rigorous proofs establish global convergence under mild assumptions and local
  superlinear convergence matching quasi-Newton methods.
\item
  \textbf{Practical Simplicity}: The algorithm requires no problem-specific hyperparameters and admits a concise
  implementation, making it accessible to practitioners.
\item
  \textbf{Problem-Specific Excellence}: QQN variants dominate on non-convex smooth problems while remaining competitive
  on convex problems, providing a robust general-purpose optimizer.
\end{enumerate}

The simplicity of QQN's core insight---that quadratic interpolation provides the natural geometry for combining
optimization directions---contrasts with the complexity of recent developments. This raises questions about the value of
geometric intuition in algorithm design.

\textbf{Note on Stochastic Extensions}: While this work focuses on deterministic optimization where line search along the
quadratic path is well-defined, stochastic extensions are conceptually feasible through ensemble sub-functional models
with deterministic seeding. Such approaches would maintain a deterministic view of the objective for path optimization
while incorporating stochastic gradient information. However, this represents a substantial extension beyond the current
scope and merits dedicated investigation.

\textbf{Computational Complexity}: The computational complexity of QQN closely mirrors that of L-BFGS, as the quadratic path
construction adds only O(n) operations to the standard L-BFGS iteration. Wall-clock time comparisons on our benchmark
problems would primarily reflect implementation details rather than algorithmic differences. For problems where function
evaluation dominates computation time, QQN's additional overhead is negligible. The geometric insights provided by
counting function evaluations offer more meaningful algorithm characterization than hardware-dependent timing
measurements.

All code, data, and results are available at \href{https://github.com/SimiaCryptus/qqn-optimizer/}{github.com/SimiaCryptus/qqn-optimizer} to
ensure reproducibility and enable further research. We encourage the community to build upon this work and explore the
broader potential of interpolation-based optimization methods.
\textbf{Note on Algorithm History}: While this paper presents the first comprehensive academic evaluation of QQN, the core
algorithm was originally developed in 2017 as documented at
\url{https://blog.simiacrypt.us/posts/optimization_research/#quadratic-quasi-newton-optimization}. The original implementation
in Java is available at \url{https://github.com/SimiaCryptus/MindsEye/blob/master/src/main/java/com/simiacryptus/mindseye/opt/orient/QQN.java}.

The quadratic interpolation principle demonstrates how geometric approaches can provide
effective solutions to optimization problems. We hope this work encourages further
exploration of simple geometric methods in optimization.

\hypertarget{acknowledgments}{%
\subsection{Acknowledgments}\label{acknowledgments}}

The QQN algorithm was originally developed and implemented by the author in 2017
(\url{https://blog.simiacrypt.us/posts/optimization_research/#quadratic-quasi-newton-optimization}), with this paper representing its first formal
academic documentation. The original Java implementation is available at
\url{https://github.com/SimiaCryptus/MindsEye/blob/master/src/main/java/com/simiacryptus/mindseye/opt/orient/QQN.java}.
AI language models assisted in the preparation of documentation, implementation of the
benchmarking framework, and drafting of the manuscript. This collaborative approach between human expertise
and AI assistance facilitated the academic presentation of the method.

\hypertarget{references}{%
\subsection{References}\label{references}}

Armijo, L. (1966). Minimization of functions having Lipschitz continuous first partial derivatives. \emph{Pacific Journal of
Mathematics}, 16(1), 1-3. https://doi.org/10.2140/pjm.1966.16.1

Beiranvand, V., Hare, W., \& Lucet, Y. (2017). Best practices for comparing optimization algorithms. \emph{Optimization and
Engineering}, 18(4), 815-848. https://doi.org/10.1007/s11081-017-9366-1

Broyden, C. G. (1970). The convergence of a class of double-rank minimization algorithms 1. General considerations. \emph{IMA
Journal of Applied Mathematics}, 6(1), 76-90. https://doi.org/10.1093/imamat/6.1.76

Cauchy, A. (1847). Méthode générale pour la résolution des systèmes d'équations simultanées. \emph{Comptes Rendus de
l'Académie des Sciences}, 25, 536-538.

De Jong, K. A. (1975). \emph{An analysis of the behavior of a class of genetic adaptive systems} (Doctoral dissertation).
University of Michigan, Ann Arbor, MI.

Fletcher, R. (1970). A new approach to variable metric algorithms. \emph{The Computer Journal}, 13(3),
317-322. https://doi.org/10.1093/comjnl/13.3.317

Goldfarb, D. (1970). A family of variable-metric methods derived by variational means. \emph{Mathematics of Computation}, 24(
109), 23-26. https://doi.org/10.1090/S0025-5718-1970-0258249-6

Hansen, N., Auger, A., Ros, R., Mersmann, O., Tušar, T., \& Brockhoff, D. (2016). COCO: A platform for comparing
continuous optimizers in a black-box setting. \emph{arXiv preprint arXiv:
1603.08785}. https://doi.org/10.48550/arXiv.1603.08785

Irwin, A., Bose, N., \& Dattani, N. S. (2020). Secant penalized BFGS: A noise robust quasi-Newton method via penalizing
the secant condition. \emph{arXiv preprint arXiv:2010.01275}. https://doi.org/10.48550/arXiv.2010.01275

Jamil, M., \& Yang, X. S. (2013). A literature survey of benchmark functions for global optimisation problems.
\emph{International Journal of Mathematical Modelling and Numerical Optimisation}, 4(2),
150-194. https://doi.org/10.1504/IJMMNO.2013.055204

Kingma, D. P., \& Ba, J. (2015). Adam: A method for stochastic optimization. \emph{Proceedings of the 3rd International
Conference on Learning Representations (ICLR)}. https://doi.org/10.48550/arXiv.1412.6980

Liang, J. J., Qu, B. Y., Suganthan, P. N., \& Hernández-Díaz, A. G. (2013). Problem definitions and evaluation criteria
for the CEC 2013 special session on real-parameter optimization. \emph{Technical Report 201212}, Computational Intelligence
Laboratory, Zhengzhou University, China.

Liu, D. C., \& Nocedal, J. (1989). On the limited memory BFGS method for large scale optimization. \emph{Mathematical
Programming}, 45(1), 503-528. https://doi.org/10.1007/BF01589116

Morales, J. L., \& Nocedal, J. (2000). Automatic preconditioning by limited memory quasi-Newton updating. \emph{SIAM Journal
on Optimization}, 10(4), 1079-1096. https://doi.org/10.1137/S1052623497327854

Moré, J. J., \& Sorensen, D. C. (1983). Computing a trust region step. \emph{SIAM Journal on Scientific and Statistical
Computing}, 4(3), 553-572. https://doi.org/10.1137/0904038

Nesterov, Y. (1983). A method for unconstrained convex minimization problem with the rate of convergence O(1/k²).
\emph{Doklady AN USSR}, 269, 543-547.

Nocedal, J., \& Wright, S. J. (2006). \emph{Numerical optimization} (2nd ed.).
Springer. https://doi.org/10.1007/978-0-387-40065-5

Polyak, B. T. (1964). Some methods of speeding up the convergence of iteration methods. \emph{USSR Computational Mathematics
and Mathematical Physics}, 4(5), 1-17. https://doi.org/10.1016/0041-5553(64)90137-5

Schmidt, M., Le Roux, N., \& Bach, F. (2021). Minimizing finite sums with the stochastic average gradient. \emph{Mathematical
Programming}, 162(1), 83-112. https://doi.org/10.1007/s10107-016-1030-6

Shanno, D. F. (1970). Conditioning of quasi-Newton methods for function minimization. \emph{Mathematics of Computation}, 24(
111), 647-656. https://doi.org/10.1090/S0025-5718-1970-0274029-X

Wolfe, P. (1969). Convergence conditions for ascent methods. \emph{SIAM Review}, 11(2),
226-235. https://doi.org/10.1137/1011036

Wolfe, P. (1971). Convergence conditions for ascent methods. II: Some corrections. \emph{SIAM Review}, 13(2),
185-188. https://doi.org/10.1137/1013035

\hypertarget{supplementary-material}{%
\subsection{Supplementary Material}\label{supplementary-material}}

Complete theorem proofs, additional experimental results, implementation details, and extended statistical analyses are
available in the supplementary material at https://github.com/SimiaCryptus/qqn-optimizer/.

\hypertarget{competing-interests}{%
\subsection{Competing Interests}\label{competing-interests}}

The authors declare no competing interests.

\hypertarget{data-availability}{%
\subsection{Data Availability}\label{data-availability}}

All experimental data, including raw optimization trajectories and statistical analyses, are available
at \href{https://github.com/SimiaCryptus/qqn-optimizer/}{github.com/SimiaCryptus/qqn-optimizer}.