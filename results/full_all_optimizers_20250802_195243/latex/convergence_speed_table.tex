\documentclass{article}
\usepackage{booktabs}
\usepackage{array}
\usepackage[table]{xcolor}
\usepackage{siunitx}
\usepackage{adjustbox}
\usepackage[margin=1in]{geometry}
\begin{document}
\begin{table}[htbp]
\centering
\caption{Convergence Speed Analysis: Mean Iterations to Reach Improvement Milestones}
\label{tab:convergence_speed}
\adjustbox{width=\textwidth,center}{
\begin{tabular}{p{3cm}p{2cm}p{2cm}p{2cm}}
\toprule
\textbf{Optimizer} & \textbf{Mean Function Evals} & \textbf{Mean Function Evals} & \textbf{Final Convergence} \\
 & \textbf{to 50\% Improvement} & \textbf{to 90\% Improvement} & \textbf{Function Evals} \\
\midrule
\textbf{QQN-GoldenSection} & 1.4 & 3.9 & 11.6 \\
L-BFGS & 2.7 & 3.5 & 12.1 \\
L-BFGS-Aggressive & 1.3 & 3.4 & 14.2 \\
\textcolor{green!70!black}{QQN-Bisection-1} & 1.4 & 4.4 & 15.7 \\
\textcolor{green!70!black}{QQN-Bisection-2} & 1.9 & 5.7 & 14.5 \\
\textcolor{green!70!black}{QQN-StrongWolfe} & 1.7 & 4.2 & 19.1 \\
\textcolor{green!70!black}{QQN-CubicQuadraticInterpolation} & 2.2 & 5.6 & 23.7 \\
L-BFGS-MoreThuente & 3.4 & 6.9 & 30.0 \\
L-BFGS-Limited & 4.4 & 8.5 & 44.3 \\
GD-AdaptiveMomentum & 11.6 & 19.1 & 24.4 \\
GD-Momentum & 10.3 & 20.8 & 37.6 \\
L-BFGS-Conservative & 7.7 & 19.2 & 56.3 \\
Adam-Fast & 15.9 & 27.4 & 47.0 \\
GD-Nesterov & 22.2 & 31.3 & 49.9 \\
Adam-Robust & 13.5 & 32.0 & 82.1 \\
Trust Region-Aggressive & 45.3 & 81.8 & 91.0 \\
GD-WeightDecay & 12.6 & 29.7 & 292.4 \\
GD & 26.1 & 71.6 & 257.5 \\
Trust Region-Standard & 77.6 & 139.8 & 155.5 \\
Trust Region-Precise & 127.3 & 229.0 & 255.4 \\
Trust Region-Adaptive & 198.2 & 356.7 & 396.8 \\
Trust Region-Conservative & 209.6 & 377.4 & 419.8 \\
Adam-AMSGrad & 138.2 & 319.8 & 578.7 \\
Adam-WeightDecay & 127.7 & 287.1 & 654.1 \\
Adam & 164.7 & 373.4 & 708.3 \\
\bottomrule
\end{tabular}
}
\end{table}
\textbf{Purpose:} Compares convergence rates for different optimizers based on total function evaluations (function + gradient evaluations). Sorted by fastest overall convergence (weighted average). Best performer is highlighted in bold, QQN variants in green.
\end{document}
